\chapter{Natural Language Processing}
\thispagestyle{fancy}
\label{chap:Natural Language Processing}

Notizen:
\begin{itemize}
	\item Natural Language Processing definieren
	\item Umfang der Anwendungsgebiete andeuten
	\item Siehe Abstract im Exposé
\end{itemize}


\section{Vorverarbeitung}
Notizen:
\begin{itemize}
	\item Pipeline der Vorverarbeitung als Voraussetzung hervorstellen
\end{itemize}


\subsection{Textbereinigung}
Notizen:


\subsection{Tokenisierung}
Notizen:


\subsection{POS-Tagging}
Notizen:


\subsection{Lemmatisierung}
Notizen:


\subsection{Entfernen von Stoppwörtern}
Notizen:


\section{Merkmalsextraktion}
Notizen:
\begin{itemize}
	\item Relevanz für extraktiven Ansatz beschreiben (vgl. Paper: „Automatic Text Summarization“)
	\item Relevanz für abstraktiven Ansatz, falls vorhanden, beschreiben
	\item Metriken selbst weiterentwickeln und ausreifen
	\item Siehe: \url{https://scikit-learn.org/stable/modules/feature_extraction.html}
\end{itemize}


\subsection{Übereinstimmung mit dem Titel}
Notizen:


\subsection{Satzposition}
Notizen:


\subsection{Satzähnlichkeit}
Notizen:


\subsection{Satzlänge}
Notizen:


\subsection{Domänenspezifische Wörter}
Notizen:


\subsection{Eigennamen}
Notizen:


\subsection{Numerische Daten}
Notizen:
