\chapter{Extraktiver Ansatz}
\thispagestyle{fancy}
\label{chap:Extraktiver Ansatz}

Notizen:
\begin{itemize}
	\item Ansatz beschreiben (vgl. Paper: „Automatic Text Summarization“)
	\item Referenzierte Ansätze nochmal aufgreifen, kritisieren und begründen, warum meine Architektur besser ist (z.B. basierten bisherige Modelle stark auf Feature Engineering, der extraktive Ansatz ist allerdings „data-driven“ (vgl. Paper: „Extractive Text Summarization using Neural Networks“)
	\item Siehe Abstract im Exposé
\end{itemize}


\section{Architektur}
Notizen:
\begin{itemize}
	\item Netzwerk des extraktiven Ansatzes als Pipeline skizzieren
	\item Bedingung “highly scalable” beschreiben und erfüllen (vgl. Paper: „Extractive Text Summarization using Neural Networks“)
	\item Sentences basierend auf Scores kopieren und ggf. neu ausrichten (vgl. Paper: „Automatic Text Summarization with Machine Learning“ und „Automatic Text Summarization Made Simple with Python“)
\end{itemize}


\section{Konfiguration}
Notizen:


\section{Training}
Notizen:
\begin{itemize}
	\item Training verschiedener Modelle
\end{itemize}


\section{Evaluation}
Notizen:
\begin{itemize}
	\item Kompressionsrate messen
	\item Qualität der Zusammenfassung messen
	\item Evaluation verschiedener Modelle mit geeigneter Vergleichstabelle
	\item Vergleich mit SOTA-Modellen
	\item Letztendlich soll das Modell über ein Skript berechnet werden, folglich soll eine Klasse entstehen, die einen Text und ein Modell einliest und eine Zusammenfassung ausgibt (Anforderung: inhaltlich und grammatikalisch korrekt)
\end{itemize}
