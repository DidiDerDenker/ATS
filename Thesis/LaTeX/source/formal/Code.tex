\chapter*{Quellcode}
\addcontentsline{toc}{chapter}{Quellcode}
\markboth{Quellcode}{}

\section*{Konfigurationsdatei}
\begin{lstlisting}[language=Python, caption=Konfigurationsdatei]
language: str = "english"  # english, german, multilingual
model_name: str = "bert-base-multilingual-cased"
tokenizer_name: str = "bert-base-multilingual-cased"
batch_size: int = 16

ratio_corpus_wik: float = 1.0
ratio_corpus_nws: float = 1.0
ratio_corpus_mls: float = 1.0
ratio_corpus_eng: float = 1.0

path_output: str = "/scratch/ws/1/davo557d-ws_project/"
path_checkpoint: str = "/scratch/ws/1/davo557d-ws_project/checkpoint-100000"

train_size: float = 0.900
val_size: float = 0.025
test_size: float = 0.075

'''
- bert-base-multilingual-cased
- deepset/gbert-base
- xlm-roberta-base
- facebook/mbart-large-cc25
'''
\end{lstlisting}
\newpage


\section*{Hilfsmethoden}
\begin{lstlisting}[language=Python, caption=Hilfsmethoden]
# Imports
import csv
import datasets
import gc
import matplotlib.pyplot as plt
import psutil
import pandas as pd
import string
import torch
import transformers
import tf2tf_gpu_config as config

from collections import Counter
from datasets import ClassLabel
from nltk import ngrams
from typing import List, Tuple


# Methods
def load_data() -> Tuple[datasets.Dataset, datasets.Dataset, datasets.Dataset]:
    if config.language == "english":
        return load_english_data()

    if config.language == "german":
        return load_german_data()

    if config.language == "multilingual":
        return load_multilingual_data()






def load_english_data()
	-> Tuple[datasets.Dataset, datasets.Dataset, datasets.Dataset]:
    train_data = datasets.load_dataset(
        "cnn_dailymail", "3.0.0", split="train",
        ignore_verifications=True
    )

    val_data = datasets.load_dataset(
        "cnn_dailymail", "3.0.0", split="validation[:50%]",
        ignore_verifications=True
    )

    test_data = datasets.load_dataset(
        "cnn_dailymail", "3.0.0", split="test[:50%]",
        ignore_verifications=True
    )

    train_data = train_data.select(
        range(0, int(len(train_data) * config.ratio_corpus_eng))
    )

    train_data = train_data.rename_column("article", "text")
    train_data = train_data.rename_column("highlights", "summary")
    train_data = train_data.remove_columns("id")

    val_data = val_data.rename_column("article", "text")
    val_data = val_data.rename_column("highlights", "summary")
    val_data = val_data.remove_columns("id")

    test_data = test_data.rename_column("article", "text")
    test_data = test_data.rename_column("highlights", "summary")
    test_data = test_data.remove_columns("id")

    return train_data.shuffle(), val_data.shuffle(), test_data.shuffle()


def load_german_data()
	-> Tuple[datasets.Dataset, datasets.Dataset, datasets.Dataset]:
    ds_wik = load_corpus_wik()
    ds_nws = load_corpus_nws()
    ds_mls = load_corpus_mls()
    
    german_data = datasets.concatenate_datasets([
        ds_wik.select(
            range(0, int(len(ds_wik) * config.ratio_corpus_wik))),
        ds_nws.select(
            range(0, int(len(ds_nws) * config.ratio_corpus_nws))),
        ds_mls.select(
            range(0, int(len(ds_mls) * config.ratio_corpus_mls)))
    ])

    train_size = int(len(german_data) * config.train_size)
    valid_size = int(len(german_data) * config.val_size)
    test_size = int(len(german_data) * config.test_size)

    train_data = german_data.select(
        range(0, train_size)
    )

    val_data = german_data.select(
        range(train_size, train_size + valid_size)
    )

    test_data = german_data.select(
        range(train_size + valid_size, train_size + valid_size + test_size)
    )

    return train_data, val_data, test_data




def load_corpus_wik() -> datasets.Dataset:
    data_txt, data_ref = [], []

    with open("./data_train.csv", "r", encoding="utf-8") as f:
        reader = csv.reader(f, delimiter=",", quoting=csv.QUOTE_ALL)
        next(reader, None)

        for row in reader:
            data_txt.append(row[0])
            data_ref.append(row[1])

    df_wik = pd.DataFrame(
        list(zip(data_txt, data_ref)),
        columns=["text", "summary"]
    )

    ds_wik = datasets.arrow_dataset.Dataset.from_pandas(df_wik)

    return ds_wik.shuffle()


def load_corpus_nws() -> datasets.Dataset:
    df_nws = pd.read_excel("./data_train_test.xlsx", engine="openpyxl")
    df_nws = df_nws[["article", "highlights"]]
    df_nws.columns = ["text", "summary"]
    df_nws = df_nws[~df_nws["summary"].str.contains("ZEIT")]
    df_nws = df_nws.dropna()
    ds_nws = datasets.arrow_dataset.Dataset.from_pandas(df_nws)
    ds_nws = ds_nws.remove_columns("__index_level_0__")

    return ds_nws.shuffle()





def load_corpus_mls() -> datasets.Dataset:
    ds_mls = datasets.load_dataset("mlsum", "de", split="train")
    ds_mls = ds_mls.remove_columns(["topic", "url", "title", "date"])

    text_corpus_mls = []
    summary_corpus_mls = []

    for entry in ds_mls:
        text = entry["text"]
        summary = entry["summary"]
        
        if summary in text:
            text = text[len(summary) + 1:len(text)]

        text_corpus_mls.append(text)
        summary_corpus_mls.append(summary)

    df_mls = pd.DataFrame(
        list(zip(text_corpus_mls, summary_corpus_mls)),
        columns=["text", "summary"]
    )

    ds_mls = datasets.arrow_dataset.Dataset.from_pandas(df_mls)

    return ds_mls.shuffle()











def load_multilingual_data()
	-> Tuple[datasets.Dataset, datasets.Dataset, datasets.Dataset]:
    english_data, _, _ = load_english_data()
    german_data, _, _ = load_german_data()

    multilingual_data = datasets.concatenate_datasets([
        german_data, english_data
    ]).shuffle()

    train_size = int(len(multilingual_data) * config.train_size)
    valid_size = int(len(multilingual_data) * config.val_size)
    test_size = int(len(multilingual_data) * config.test_size)

    train_data = multilingual_data.select(
        range(0, train_size)
    )

    val_data = multilingual_data.select(
        range(train_size, train_size + valid_size)
    )
    
    test_data = multilingual_data.select(
        range(train_size + valid_size, train_size + valid_size + test_size)
    )

    return train_data, val_data, test_data


def test_cuda() -> None:
    device = torch.device("cuda" if torch.cuda.is_available() else "cpu")
    torch.cuda.empty_cache()

    print("Device:", device)
    print("Version:", torch.__version__)


def explore_corpus(data: datasets.Dataset) -> None:
    df = pd.DataFrame(data)

    text_list = []
    summary_list = []

    for index, row in df.iterrows():
        text = row["text"]
        summary = row["summary"]
        text_list.append(len(text))
        summary_list.append(len(summary))

    df = pd.DataFrame(data[:1])

    for column, typ in data.features.items():
        if isinstance(typ, ClassLabel):
            df[column] = df[column].transform(lambda i: typ.names[i])


def empty_cache() -> None:
    gc.collect()
    torch.cuda.empty_cache()
    psutil.virtual_memory()













def load_tokenizer_and_model(from_checkpoint: bool = False)
	-> Tuple[transformers.AutoTokenizer, transformers.EncoderDecoderModel]:
    tokenizer = transformers.AutoTokenizer.from_pretrained(
        config.tokenizer_name, strip_accent=False
    )

    if from_checkpoint:
        if "mbart" in config.model_name:
            tf2tf = transformers.AutoModelForSeq2SeqLM.from_pretrained(
                config.path_checkpoint
            )

        else:
            tf2tf = transformers.EncoderDecoderModel.from_pretrained(
                config.path_checkpoint
            )

    else:
        if "mbart" in config.model_name:
            tf2tf = transformers.MBartForConditionalGeneration.from_pretrained(
                config.model_name
            )

        else:
            tf2tf = transformers.EncoderDecoderModel.from_encoder_decoder_pretrained(
                config.model_name, config.model_name, tie_encoder_decoder=True
            )

    return tokenizer, tf2tf
    
    
    
    
    
    
def configure_model(tf2tf: transformers.EncoderDecoderModel, tokenizer: transformers.AutoTokenizer)
	-> transformers.EncoderDecoderModel:
    tf2tf.config.decoder_start_token_id = tokenizer.cls_token_id
    tf2tf.config.bos_token_id = tokenizer.bos_token_id
    tf2tf.config.eos_token_id = tokenizer.sep_token_id
    tf2tf.config.pad_token_id = tokenizer.pad_token_id

    tf2tf.config.max_length = 128
    tf2tf.config.min_length = 56
    tf2tf.config.no_repeat_ngram_size = 3
    tf2tf.config.early_stopping = True
    tf2tf.config.length_penalty = 2.0
    tf2tf.config.num_beams = 2

    return tf2tf
\end{lstlisting}
\newpage


\section*{Trainingscode}
\begin{lstlisting}[language=Python, caption=Trainingscode]
# Imports
import datasets
import transformers
import tf2tf_gpu_config as config
import tf2tf_gpu_helpers as helpers


# Main
tokenizer, tf2tf = helpers.load_tokenizer_and_model(from_checkpoint=False)
train_data, val_data, test_data = helpers.load_data()
rouge = datasets.load_metric("rouge")

helpers.test_cuda()
helpers.explore_corpus(train_data)
helpers.empty_cache()

tf2tf = helpers.configure_model(tf2tf, tokenizer)
tf2tf.to("cuda")
















def process_data_to_model_inputs(batch):
    encoder_max_length = 512
    decoder_max_length = 128

    inputs = tokenizer(batch["text"], padding="max_length",
                       truncation=True, max_length=encoder_max_length)

    outputs = tokenizer(batch["summary"], padding="max_length",
                        truncation=True, max_length=decoder_max_length)

    batch["input_ids"] = inputs.input_ids
    batch["attention_mask"] = inputs.attention_mask
    batch["decoder_input_ids"] = outputs.input_ids
    batch["decoder_attention_mask"] = outputs.attention_mask
    batch["labels"] = outputs.input_ids.copy()
    batch["labels"] = [[-100 if token == tokenizer.pad_token_id else token
    									for token in labels] for labels in batch["labels"]]

    return batch


train_data = train_data.map(
    process_data_to_model_inputs,
    batched=True,
    batch_size=config.batch_size,
    remove_columns=["text", "summary"]
)

train_data.set_format(
    type="torch",
    columns=["input_ids",
             "attention_mask",
             "decoder_input_ids",
             "decoder_attention_mask",
             "labels"]
)
val_data = val_data.map(
    process_data_to_model_inputs,
    batched=True,
    batch_size=config.batch_size,
    remove_columns=["text", "summary"]
)

val_data.set_format(
    type="torch",
    columns=["input_ids",
             "attention_mask",
             "decoder_input_ids",
             "decoder_attention_mask",
             "labels"]
)


def compute_metrics(pred):
    labels_ids = pred.label_ids
    pred_ids = pred.predictions

    pred_str = tokenizer.batch_decode(pred_ids, skip_special_tokens=True)
    labels_ids[labels_ids == -100] = tokenizer.pad_token_id
    label_str = tokenizer.batch_decode(labels_ids, skip_special_tokens=True)

    rouge_output = rouge.compute(
        predictions=pred_str,
        references=label_str,
        rouge_types=["rouge2"]
    )["rouge2"].mid

    return {
        "rouge2_precision": round(rouge_output.precision, 4),
        "rouge2_recall": round(rouge_output.recall, 4),
        "rouge2_fmeasure": round(rouge_output.fmeasure, 4),
    }
if "mbart" in config.model_name:
    training_args = transformers.TrainingArguments(
        output_dir=config.path_output,
        logging_dir=config.path_output,
        per_device_train_batch_size=1,
        per_device_eval_batch_size=1,
        num_train_epochs=1,
        warmup_steps=500,
        weight_decay=0.01
    )

    trainer = transformers.Trainer(
        model=tf2tf,
        args=training_args,
        train_dataset=train_data,
        eval_dataset=val_data
    )

else:
    training_args = transformers.Seq2SeqTrainingArguments(
        predict_with_generate=True,
        evaluation_strategy="steps",
        per_device_train_batch_size=config.batch_size,
        per_device_eval_batch_size=config.batch_size,
        output_dir=config.path_output,
        warmup_steps=1000,
        save_steps=10000,
        logging_steps=2000,
        eval_steps=10000,
        save_total_limit=1,
        learning_rate=5e-5,
        adafactor=True,
        fp16=True
    )


    trainer = transformers.Seq2SeqTrainer(
        model=tf2tf,
        args=training_args,
        compute_metrics=compute_metrics,
        train_dataset=train_data,
        eval_dataset=val_data,
        tokenizer=tokenizer
    )

trainer.train()
\end{lstlisting}
\newpage


\section*{Evaluationscode}
\begin{lstlisting}[language=Python, caption=Evaluationscode]
# Imports
import datasets
import transformers
import tf2tf_gpu_config as config
import tf2tf_gpu_helpers as helpers


# Main
rouge = datasets.load_metric("rouge")

if "mbart" in config.model_name:
    data, _, _ = helpers.load_data()
    model = transformers.pipeline("summarization")
    
    list_of_texts = list(data["text"])
    list_of_candidates = list(data["summary"])
    list_of_summaries, list_of_predictions = [], []   

    for i in range(0, len(list_of_texts) - 1):
        try:
            summary = model(
                str(list_of_texts[i])[0:1024],
                max_length=156,
                min_length=64,
                do_sample=False
            )[0]["summary_text"]

            list_of_predictions.append(summary)
            list_of_summaries.append(list_of_candidates[i])

        except Exception as e:
            print(e)


    print(
        rouge.compute(
            predictions=list_of_predictions,
            references=list_of_summaries,
            rouge_types=["rouge2"]
        )["rouge2"].mid
    )

else:
    tokenizer, tf2tf = helpers.load_tokenizer_and_model(from_checkpoint=True)
    train_data, val_data, test_data = helpers.load_data()

    tf2tf = helpers.configure_model(tf2tf, tokenizer)
    tf2tf.to("cuda")

    def generate_summary(batch):
        inputs = tokenizer(
            batch["text"],
            padding="max_length",
            truncation=True,
            max_length=512,
            return_tensors="pt"
        )

        input_ids = inputs.input_ids.to("cuda")
        attention_mask = inputs.attention_mask.to("cuda")

        outputs = tf2tf.generate(input_ids, attention_mask=attention_mask)
        output_str = tokenizer.batch_decode(outputs, skip_special_tokens=True)
        batch["pred_summary"] = output_str

        return batch




    results = test_data.map(
        generate_summary,
        batched=True,
        batch_size=config.batch_size
    )

    print(
        rouge.compute(
            predictions=results["pred_summary"],
            references=results["summary"],
            rouge_types=["rouge2"]
        )["rouge2"].mid
    )
\end{lstlisting}
\newpage


\section*{Beispielcode}
\begin{lstlisting}[language=Python, caption=Beispielcode]
# Installation
%%capture

!pip install transformers==4.5.1
!pip install datasets==1.6.2
!pip install tokenizers==0.10.2
!pip install torch==1.8.1+cu111
!pip install psutil==5.8.0
!pip install rouge_score
!pip install sacrebleu
!pip install openpyxl
!pip install xlrd
!pip install git-python
!pip install -U ipython==7.20
!pip install cmake
!pip install SentencePiece


# Imports
import csv
import datasets
import gc
import psutil
import pandas as pd
import torch
import transformers

from datasets import ClassLabel
from IPython.display import display, HTML
from typing import Tuple




# Drive
from google.colab import drive
drive.mount("/content/drive")
path_drive = "/content/drive/My Drive/Temp/"


# Methods
def split_long_texts(parts: List[str], text: str):
    limit = 512

    if len(text) > limit:
        end_index = max([
            text.rfind(".", 0, limit),
            text.rfind("!", 0, limit),
            text.rfind("?", 0, limit)
        ])

        parts.append(text[0:end_index + 1].strip())
        text = text[end_index + 1:len(text)].strip()
        parts = split_long_texts(parts, text)

    else:
        parts.append(text)

    return parts
    










# Example (BERT, XLM-R)
tokenizer, tf2tf = load_tokenizer_and_model(from_checkpoint=True)

if language == "german":
    corpus = corpus_german

if language == "english":
    corpus = corpus_english

if language == "multilingual":
    corpus = corpus_english + corpus_german

tf2tf = configure_model(tf2tf, tokenizer)
tf2tf.to("cuda")

test_cuda()
empty_cache()

cnt = 0

for text in corpus:
    cnt += 1
    parts = split_long_texts([], text)

    if len(parts) > 1:
        article = parts
        highlights = [None] * len(parts)

    else:
        parts = [text]
        article = [text] * 2
        highlights = [None] * 2

    df = pd.DataFrame({"text": article, "summary": highlights})
    test_data = datasets.arrow_dataset.Dataset.from_pandas(df)

    def generate_summary(batch):
        inputs = tokenizer(
            batch["text"],
            padding="max_length",
            truncation=True,
            max_length=512,
            return_tensors="pt"
        )
        
        input_ids = inputs.input_ids.to("cuda")
        attention_mask = inputs.attention_mask.to("cuda")

        outputs = tf2tf.generate(input_ids, attention_mask=attention_mask)
        output_str = tokenizer.batch_decode(outputs, skip_special_tokens=True)
        batch["pred_summary"] = output_str

        return batch


    summary = test_data.map(
        generate_summary,
        batched=True,
        batch_size=batch_size
    )

    result = ""

    for i in range(0, len(parts)):
        result = result + " " + summary[i]["pred_summary"]

    with open(path_output + "summary_" + language + "_" + model_name + "_text-" + str(cnt) + ".txt", "w", encoding="utf-8") as f:
        f.write(summary[0]["pred_summary"])



# Example (BART)
model = transformers.pipeline("summarization")
corpus = corpus_english + corpus_german

cnt = 0

for text in corpus:
    cnt += 1
    
    summary = model(
        text,
        max_length=156,
        min_length=64,
        do_sample=False
    )[0]["summary_text"]

    with open(path_output + "summary_" + language + "_BART_text-" + str(cnt) + ".txt", "w", encoding="utf-8") as f:
        f.write(summary)
\end{lstlisting}
\newpage

\section*{Datenexploration}
\begin{lstlisting}[language=Python, caption=Datenexploration]
# Imports
import datasets
import transformers
import tf2tf_gpu_config as config
import tf2tf_gpu_helpers as helpers


# Methods
def clean(text: str) -> str:
    text = " ".join([w.lower() for w in text.split()])
    text = text.translate(str.maketrans("", "", string.punctuation))

    return text


def analyze_text_lenghts(corpus: List[str]) -> None:
    lengths = []

    for text in corpus:
        lengths.append(len(text.split()))

    fig = plt.figure(figsize=(10, 6))

    plt.hist(lengths, bins=40)
    plt.xlabel("Anzahl der Woerter", fontsize=18)
    plt.ylabel("Anzahl der Texte", fontsize=18)
    plt.xticks(rotation=0, fontsize=12)
    plt.yticks(rotation=0, fontsize=12)
    plt.tight_layout()

    plt.show()



def analyze_most_common_words(corpus: List[str], k: int = 20) -> None:
    words = []

    for text in corpus:
        token_list = clean(text).split()

        for token in token_list:
            if len(token) > 1:
                words.append(token)

    most_common_words = Counter(words).most_common(k)

    x = [tuple[0] for tuple in most_common_words]
    y = [tuple[1] for tuple in most_common_words]

    fig = plt.figure(figsize=(10, 6))

    plt.bar(x, y)
    plt.xlabel("Woerter", fontsize=18)
    plt.ylabel("Anzahl", fontsize=18)
    plt.xticks(rotation=90, fontsize=12)
    plt.yticks(rotation=0, fontsize=12)
    plt.tight_layout()

    plt.show()


def analyze_n_grams(corpus: List[str], n: int = 3, n_most_common_n_grams: int = 20) -> None:
    n_grams = []

    for text in corpus:
        try:
            for n_gram in ngrams(text.split(), n):
                n_grams.append(n_gram)

        except Exception as e:
            print(e)

    most_common_n_grams = Counter(n_grams).most_common(n_most_common_n_grams)

    x = [" ".join(tuple[0]) for tuple in most_common_n_grams]
    y = [tuple[1] for tuple in most_common_n_grams]

    fig = plt.figure(figsize=(10, 6))

    plt.bar(x, y)
    plt.xlabel("N-Gramme", fontsize=18)
    plt.ylabel("Anzahl", fontsize=18)
    plt.xticks(rotation=90, fontsize=12)
    plt.yticks(rotation=0, fontsize=12)
    plt.tight_layout()

    plt.show()
    
    
# Main
data, _, _ = load_data()
corpus = list(data["text"])

analyze_text_lenghts(corpus)
analyze_most_common_words(corpus)
analyze_n_grams(corpus, n=2)
analyze_n_grams(corpus, n=3)
\end{lstlisting}
