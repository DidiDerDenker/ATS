\appendix

\chapter{Qualitative Analyse (Englisch)}
\noindent
\textbf{Text 1: Bericht über die Terroranschläge des 11. Septembers 2001}\\

\noindent\fbox{%
\parbox{\textwidth}{%
The perpetrators knew what they were doing. The attacks were media-ready. The symbols they destroyed were precisely chosen: The White House and the Pentagon as symbols of power, the World Trade Center as a symbol of capital and business. 2,750 people, including 30 Germans, died in or on the towers of the WTC, 147 in the airplanes, 184 victims lost their lives in the Pentagon, 40 people were killed in the plane that crashed in Pennsylvania in the most perfidious terrorist attack yet unleashed by Islamist terrorists. The U.S. was in a state of shock: The attacks were the first quasi-military attack on the United States since 1814. America proved vulnerable at home on Sept. 11, 2001, for the first time since the struggles for its independence from England in the late 18th and early 19th centuries. As a result, the attacks had serious political and military consequences around the world. U.S. President George W. Bush, who was in Florida at the time of the attacks, made the following remarks in an initial televised address: \grqq{}Make no mistake. We will hunt these people down and punish them to the end.\grqq{}
}%
}\\
\newline

\noindent
\textbf{Zusammenfassungen von Text 1}\\

\noindent
Gold-Standard:\\

\noindent\fbox{%
\parbox{\textwidth}{%
The perpetrators destroyed symbolic buildings, the White House and the Pentagon. 2750 people died in the WTC, 147 in airplanes and 184 in the Pentagon. The terrorist attack was the first military attack on the U.S. since 1814. President Bush announced that he would punish the perpetrators.
}%
}\\
\newpage

\noindent
BERT (monolingual):\\

\noindent\fbox{%
\parbox{\textwidth}{%
The perpetrators knew what they were doing. The White House and the Pentagon as symbols of power, the World Trade Center as a symbol of capital and business. The attacks were media-ready and media-prepared. The perimeters were precisely chosen: The White House.
}%
}\\
\newline

\noindent
BERT (multilingual):\\

\noindent\fbox{%
\parbox{\textwidth}{%
The attacks were media-ready, but the perpetrators knew what they were doing. The attacks are a symbol of capital and business, the Pentagon says. The U.S. are investigating the attacks, the White House says.
}%
}\\
\newline

\noindent
XLM-R (monolingual):\\

\noindent\fbox{%
\parbox{\textwidth}{%
The attacks were media-ready, the Pentagon and the Pentagon. The attack was taken by the Pentagon as a symbol of power, the World Trade Center. The perpetrators were used by power.
}%
}\\
\newline

\noindent
XLM-R (multilingual):\\

\noindent\fbox{%
\parbox{\textwidth}{%
In the wake of the World Trade Center attacks, the U.S. has seen a huge surge in violence. The attacks are part of a global global economy. It's not the first time that the attacks occur in the United States. In the past two years, the world has seen an uncertainty.
}%
}\\
\newline

\noindent
BART (multilingual):\\

\noindent\fbox{%
\parbox{\textwidth}{%
U.S. proved vulnerable at home on Sept. 11, 2001, for the first time since independence struggles. The perpetrators knew what they were doing. The attacks were media-ready. The symbols they destroyed were precisely chosen: The White House and the Pentagon as symbols of power, the World Trade Center as symbol of capital and business.
}%
}\\
\newpage

\noindent
\textbf{Text 2: Bericht über das Finale der WM 2014}\\

\noindent\fbox{%
\parbox{\textwidth}{%
German coach Joachim Löw saw no reason to change his starting eleven after the fantastic 7-1 triumph in the semifinals against host Brazil - until shortly before kickoff. Then the coach received bad news: Khedira had to pull out at short notice with calf problems, and Kramer was replaced in the starting lineup. The DFB team tried hard, but the South Americans were more determined. After a cross from Lavezzi, Higuain put in what was supposed to be a 1-0, but was correctly whistled back for an offside position. The longer the game lasted, the more both sides shied away from taking risks. The tempo of the match, which had been so racy in the first period, also dwindled with the courage. Instead, the duels became rougher: Mascherano and Aguero received warnings. And whenever a chance presented itself unexpectedly, a technical error crept in. It remained a tough struggle. Mascherano and Aguero, who had already been penalized, were lucky not to be sent off early with a yellow card when they first knocked Schweinsteiger down and then gave him a cut to the face. Schürrle put in a spirited spurt and found Götze in the center, who picked the ball off his chest and expertly fired in a left-footed volley - a dream goal at a very important time.
}%
}\\
\newline

\noindent
\textbf{Zusammenfassungen von Text 2}\\

\noindent
Gold-Standard:\\

\noindent\fbox{%
\parbox{\textwidth}{%
Joachim Löw did not change his starting lineup after the 7-1 win against Brazil. Shortly before the game, he had to replace Khedira with Kramer. Higuain scored for Argentina but was offside. The teams took fewer and fewer risks. Two players from Argentina were lucky not to get a sending off. Götze decided the game with a dream goal.
}%
}\\
\newpage

\noindent
BERT (monolingual):\\

\noindent\fbox{%
\parbox{\textwidth}{%
Joachim Löw saw no reason to change his starting eleven. Khedira had to pull out at short notice with calf problems. He was replaced in the starting line-up by Kramer. Kramer was replaced by Kramer in the first half. The DFB team tried hard but the South Americans were more determined.
}%
}\\
\newline

\noindent
BERT (multilingual):\\

\noindent\fbox{%
\parbox{\textwidth}{%
Joachim Löw saw no reason to change his starting eleven after 7-1 win over Brazil. The DFB team tried hard to change their starting eleven. Khedira had to pull out at short notice with calf problems. He was replaced in the starting lineup by the South Americans.
}%
}\\
\newline

\noindent
XLM-R (monolingual):\\

\noindent\fbox{%
\parbox{\textwidth}{%
Khedira had to pull out at short notice with calf problems. Kramer replaced the coach Joachim Löw. The coach was replaced in the starting lineup. Joachim Löw replaced his starting eleven 7-1 victory against host Brazil.
}%
}\\
\newline

\noindent
XLM-R (multilingual):\\

\noindent\fbox{%
\parbox{\textwidth}{%
Joachim Löw hat die 7-1 triumph in der 7-1-Niederlage gegen Gastgeber Brazil gelassen. Joachim Löws 7-1 wins in der Champions League gegen Brazil. Joachim Gauck scheitert an der WM-Qualifikation in Brasilien.
}%
}\\
\newline

\noindent
BART (multilingual):\\

\noindent\fbox{%
\parbox{\textwidth}{%
Joachim Löw saw no reason to change his starting eleven after Germany's 7-1 win over Brazil. Mascherano and Aguero were lucky not to be sent off early with a yellow card. The tempo of the match dwindled with the courage of the South Americans. Götze opened the scoring for Germany with a left-footed volley.
}%
}\\
\newpage

\noindent
\textbf{Text 3: Definition von Inflation}\\

\noindent\fbox{%
\parbox{\textwidth}{%
Inflation is a sustained process of demonetization, which manifests itself in general price increases. One monetary unit can then constantly buy less, i.e. the purchasing power of money is permanently reduced. Inflation does not include one-time, temporary price level increases caused by unusual occurrences or price increases for specific goods or production factors. Inflation is measured by the increase in a price index that best reflects the general price level, such as the consumer price index for Germany. The percentage increase in the price index over a given period is referred to as the inflation rate. The quantity of money in the economy plays a particularly important role in the occurrence of inflation. If the quantity of goods in the economy as a whole is matched by an excessively large quantity of money, a condition for inflation is met. If the aggregate demand for goods exceeds the aggregate supply of goods, which cannot be increased in the short term, rising prices are the consequence and inflation sets in. The price increases trigger rising wages, and because of the higher income, the demand for goods increases. However, the higher wages also cause companies' costs to rise, which in turn leads to price increases for goods.
}%
}\\
\newline

\noindent
\textbf{Zusammenfassungen von Text 3}\\

\noindent
Gold-Standard:\\

\noindent\fbox{%
\parbox{\textwidth}{%
Inflation is a sustained process in which the purchasing power of money is permanently reduced. It is measured by the inflation rate. Prices rise, wages rise, and thus demand also rises. Higher wages in turn cause higher costs for companies, which is why prices for goods continue to rise.
}%
}\\
\newpage

\noindent
BERT (monolingual):\\

\noindent\fbox{%
\parbox{\textwidth}{%
Inflation manifests itself in general price increases. Inflation does not include one-time, temporary price level increases caused by unusual occurrences or price increases for specific goods or production factors. The purchasing power of money is permanently reduced.
}%
}\\
\newline

\noindent
BERT (multilingual):\\

\noindent\fbox{%
\parbox{\textwidth}{%
Inflation is a sustained process of demonetization, which manifests itself in general price increases. Inflation does not include one-time, temporary price level increases caused by unusual occurrences or price increases for specific goods or production factors. The purchasing power of money is permanently reduced.
}%
}\\
\newline

\noindent
XLM-R (monolingual):\\

\noindent\fbox{%
\parbox{\textwidth}{%
The purchasing power is a sustained process of demonetization, which manifests itself in general price increases. A sustained power of money is permanently reduced by unusual occurrences or price increase.
}%
}\\
\newline

\noindent
XLM-R (multilingual):\\

\noindent\fbox{%
\parbox{\textwidth}{%
Inflation is a sustained process of demonetization. Inflation does not include one-time, temporary price level increases. The purchasing power of money is permanently reduced. In general price increases are permanently decreased by one factor.
}%
}\\
\newline

\noindent
BART (multilingual):\\

\noindent\fbox{%
\parbox{\textwidth}{%
Inflation is a sustained process of demonetization, which manifests itself in general price increases. One monetary unit can then constantly buy less, i.e. the purchasing power of money is permanently reduced. Inflation does not include one-time, temporary price level increases caused by unusual occurrences or price increases for specific goods or production factors.
}%
}\\
\newpage

\noindent
\textbf{Text 4: Rechtsurteil zu Stornierungen während einer Pandemie}\\

\noindent\fbox{%
\parbox{\textwidth}{%
If a traveler cancels the trip due to a virus pandemic, the tour operator is not entitled to compensation under Section 651 h (1) sentence 3 of the German Civil Code (BGB) if the traveler also cancels the trip. In addition, the lack of treatment options and vaccination justifies a right to cancel the trip free of charge. This decided the district court Stuttgart. The case was based on the following facts: A few days before the start of a multi-day flight and bus tour through Portugal in March 2020, the traveler declared the cancellation. She justified this with the spreading Corona virus. The tour operator also canceled the trip two days later. Nevertheless, it demanded payment of a cancellation fee from the traveler. Since the traveler refused to comply, a lawsuit was filed. The Stuttgart District Court ruled in favor of the traveler. The tour operator was not entitled to compensation pursuant to Section 651 h (1) sentence 3 BGB. On the one hand, the traveler could invoke extraordinary circumstances within the meaning of Section 651 h (3) of the German Civil Code. Secondly, it had to be taken into account that the tour operator had also cancelled the trip.
}%
}\\
\newline

\noindent
\textbf{Zusammenfassungen von Text 4}\\

\noindent
Gold-Standard:\\

\noindent\fbox{%
\parbox{\textwidth}{%
If a traveler cancels the trip due to a virus pandemic, the tour operator is not entitled to compensation. Travelers therefore do not have to pay a cancellation fee if the tour operator has also canceled the trip.
}%
}\\
\newpage

\noindent
BERT (monolingual):\\

\noindent\fbox{%
\parbox{\textwidth}{%
Tour operator is not entitled to compensation under Section 651 HGB. The lack of treatment options and vaccination justifies a right to cancel the trip free of charge. This decided the district court Stuttgart. The case is the result of a virus pandemic in the German Civil Code.
}%
}\\
\newline

\noindent
BERT (multilingual):\\

\noindent\fbox{%
\parbox{\textwidth}{%
If a traveler cancels the trip due to a virus pandemic, the tour operator is not entitled to compensation. The lack of treatment options and vaccination justifies a right to cancel the trip. The district court Stuttgart has ruled that there is no treatment for the virus.
}%
}\\
\newline

\noindent
XLM-R (monolingual):\\

\noindent\fbox{%
\parbox{\textwidth}{%
Tour operator cancels the trip due to a virus pandemic. The tour operator is not entitled to compensation. In addition, the lack of treatment options and vaccination justifies a right to cancel the trip free of charge.
}%
}\\
\newline

\noindent
XLM-R (multilingual):\\

\noindent\fbox{%
\parbox{\textwidth}{%
If a traveler cancels the trip due to a virus pandemic, the tour operator is not entitled to compensation. If a trip is canceled, the trip operator is entitled to refund the trip free of charge. The court Stuttgart has decided the trip will be free of charges.
}%
}\\
\newline

\noindent
BART (multilingual):\\

\noindent\fbox{%
\parbox{\textwidth}{%
A traveler cancels the trip due to a virus pandemic, the tour operator is not entitled to compensation under Section 651 h (1) sentence 3 of the German Civil Code. In addition, the lack of treatment options and vaccination justifies a right to cancel the trip free of charge. The Stuttgart District Court ruled in favor of the traveler.
}%
}\\
\newpage

\noindent
\textbf{Text 5: Anleitung zum Händewaschen}\\

\noindent\fbox{%
\parbox{\textwidth}{%
Washing your hands is easy, and it’s one of the most effective ways to prevent the spread of germs. Clean hands can stop germs from spreading from one person to another and throughout an entire community. Follow these five steps every time. Wet your hands with clean, running water (warm or cold), turn off the tap, and apply soap. Lather your hands by rubbing them together with the soap. Lather the backs of your hands, between your fingers, and under your nails. Scrub your hands for at least 20 seconds. Rinse your hands well under clean, running water. Dry your hands using a clean towel or air dry them.
}%
}\\
\newline

\noindent
\textbf{Zusammenfassungen von Text 5}\\

\noindent
Gold-Standard:\\

\noindent\fbox{%
\parbox{\textwidth}{%
Hand washing is the most effective way to prevent the spread of germs. For this, you should wash your hands with cold water, then add soap, rub properly for 20 seconds and then rinse under cold water.
}%
}\\
\newpage

\noindent
BERT (monolingual):\\

\noindent\fbox{%
\parbox{\textwidth}{%
Clean hands can stop germs spreading from one person to another. Clean hands with clean, running water, and apply soap. Lather your hands by rubbing them together with soap. Scrub your hands for at least 20 seconds. Lather the backs of your hands between your fingers and under your nails.
}%
}\\
\newline

\noindent
BERT (multilingual):\\

\noindent\fbox{%
\parbox{\textwidth}{%
Washing your hands is easy, and it's one of the most effective ways to prevent the spread of germs. Clean hands can stop germs from spreading from one person to another.
}%
}\\
\newline

\noindent
XLM-R (monolingual):\\

\noindent\fbox{%
\parbox{\textwidth}{%
Clean hands can stop germs from one person to another. Clean hands with clean, running water (warm or cold), turn off the tap, and apply soap. Follow the five steps every time, you're one of the most effective ways to prevent the spread of germs.
}%
}\\
\newline

\noindent
XLM-R (multilingual):\\

\noindent\fbox{%
\parbox{\textwidth}{%
Clean hands can stop germs from spreading from one person to another. Clean hands for at least 20 seconds every time you use the soap. Follow these five steps every time.
}%
}\\
\newline

\noindent
BART (multilingual):\\

\noindent\fbox{%
\parbox{\textwidth}{%
Washing your hands is one of the most effective ways to prevent the spread of germs. Clean hands can stop germs from spreading from one person to another and throughout an entire community. Wet your hands with clean, running water (warm or cold), turn off the tap, and apply soap.
}%
}\\


\chapter{Qualitative Analyse (Deutsch)}
\noindent
\textbf{Text 1: Bericht über die Terroranschläge des 11. Septembers 2001}\\

\noindent\fbox{%
\parbox{\textwidth}{%
Die Täter wussten, was sie taten. Die Anschläge waren mediengerecht umgesetzt. Die Symbole, die sie zerstörten, waren präzise ausgewählt: Das weiße Haus und das Pentagon als Symbole der Macht, das World Trade Center als Symbol des Kapitals und der Wirtschaft. 2.750 Menschen, darunter 30 Deutsche, starben in oder an den Türmen des WTC, 147 in den Flugzeugen, 184 Opfer kamen im Pentagon ums Leben, 40 Menschen wurden in dem in Pennsylvania abgestürtzten Flugzeug Opfer des bisher perfidesten Terroranschlags, den islamistische Terroristen ausgelöst haben. Die USA befanden sich im Schockzustand: Die Anschläge waren der erste quasi militärische Angriff auf die Vereinigten Staaten seit 1814. Amerika erwies sich am 11. September 2001 erstmals seit den Kämpfen um seine Unabhängigkeit von England im späten 18. und frühen 19. Jahrhundert als im eigenen Land verwundbar. Die Anschläge hatten daher weltweit gravierende politische und militärische Folgen. Der amerikanische Präsident George W. Bush, der sich zum Zeitpunkt der Anschläge in Florida aufhielt, äußert sich in einer ersten Fernsehansprache: \grqq{}Täuschen Sie sich nicht. Wir werden diese Leute bis zum Ende jagen und bestrafen.\grqq{}
}%
}\\
\newline

\noindent
\textbf{Zusammenfassungen von Text 1}\\

\noindent
Gold-Standard:\\

\noindent\fbox{%
\parbox{\textwidth}{%
Die Täter zerstörten symbolische Gebäude, das weiße Haus und das Pentagon. 2750 Menschen starben im WTC, 147 in Flugzeugen und 184 im Pentagon. Der Terroranschlag war der erste militärische Angriff auf die USA seit 1814. Präsident Bush kündigte an, dass die Täter gejagt und bestraft werden.
}%
}\\
\newpage

\noindent
BERT (monolingual):\\

\noindent\fbox{%
\parbox{\textwidth}{%
Die Anschläge in New York und Washington zeigen, wie die Terroristen in den USA die Macht verloren haben. Eine Übersicht über die Attentäter, die die Terrormiliz bekämpfen will.
}%
}\\
\newline

\noindent
BERT (multilingual):\\

\noindent\fbox{%
\parbox{\textwidth}{%
Die Täter wussten, was sie taten: Was die Täter in den USA taten, war Zerstörung. Und wie sie die Stadt und das Pentagon als Symbole der Macht bewerten. Die Bilder der Anschläge in Berlin.
}%
}\\
\newline

\noindent
XLM-R (monolingual):\\

\noindent\fbox{%
\parbox{\textwidth}{%
Die Anschläge auf die World Trade Center waren die ersten Anschläge in den USA. Sie waren die letzten, in denen sich die USA und die USA aufeinander einlassen mussten.
}%
}\\
\newline

\noindent
XLM-R (multilingual):\\

\noindent\fbox{%
\parbox{\textwidth}{%
In den USA ist die Demokratie ein Symbol des Kapitalismus geworden. Die Bilder von der Revolution sind in den USA immer noch präzise geworden. Doch die Bilder sind längst nicht mehr so schön, wie es ist.
}%
}\\
\newline

\noindent
BART (multilingual):\\

\noindent\fbox{%
\parbox{\textwidth}{%
2.750 Menschen, darunter 30 Deutsche, starben in oder an den Türmen des WTC, 147 in den Flugzeugen, 184 Opfer kamen im Pentagon ums Leben. Die Anschläge waren der erste quasi militärische Angriff auf die Vereinigten Staaten seit 1814.
}%
}\\
\newpage

\noindent
\textbf{Text 2: Bericht über das Finale der WM 2014}\\

\noindent\fbox{%
\parbox{\textwidth}{%
Bundestrainer Joachim Löw sah nach dem famosen 7:1-Triumph im Halbfinale gegen Gastgeber Brasilien keinen Anlass, seine Startelf umzustellen - bis kurz vor Anpfiff. Dann erreichte den Bundestrainer die Hiobsbotschaft: Khedira meldete sich mit Wadenproblemen kurzfristig ab, für ihn rückte Kramer in die Startformation. Das DFB-Team agierte durchaus bemüht, doch zielstrebiger blieben die Südamerikaner. Nach einer Flanke von Lavezzi schob Higuain zum vermeintlichen 1:0 ein, wurde aber wegen einer Abseitsposition korrekterweise zurückgepfiffen. Je länger die Partie dauerte, desto mehr scheuten beide Seiten das Risiko. Mit dem Mut schwand auch das Tempo aus der im ersten Abschnitt noch so rassigen Begegnung. Dafür wurden die Zweikämpfe ruppiger: Mascherano und Aguero kassierten Verwarnungen. Und wenn sich mal unverhofft eine Chance bot, schlich sich ein technischer Fehler ein. Es blieb ein zähes Ringen. Die bereits vorbelasteten Mascherano und Aguero hatten Glück, dass sie nicht vorzeitig mit Gelb-Rot vom Platz mussten, als sie Schweinsteiger erst umsensten und dann eine Platzwunde im Gesicht verschafften. Zwei ließ das kalt: Schürrle zog noch einen beherzten Spurt an und fand Götze im Zentrum, der das Spielgerät mit der Brust runterpflückte und gekonnt mit links volley einschoss - ein Traumtor zu einem ganz wichtigen Zeitpunkt.
}%
}\\
\newline

\noindent
\textbf{Zusammenfassungen von Text 2}\\

\noindent
Gold-Standard:\\

\noindent\fbox{%
\parbox{\textwidth}{%
Joachim Löw änderte seine Startaufstellung nach dem 7:1-Sieg gegen Brasilien nicht. Kurz vor dem Spiel musste er Khedira durch Kramer ersetzen. Higuain erzielte ein Abseitstor für Argentinien. Die Mannschaften gingen im Spielverlauf immer weniger Risiko ein. Zwei argentinische Spieler hatten Glück, dass sie keinen Platzverweis bekommen haben. Götze entschied das Spiel mit einem Traumtor auf Vorlage von Schürrle.
}%
}\\
\newpage

\noindent
BERT (monolingual):\\

\noindent\fbox{%
\parbox{\textwidth}{%
Die DFB-Elf verliert im Halbfinale gegen Brasilien mit 7:1. Das Team von Joachim Löw zeigt, wie sehr sich das Team von Jürgen Klopp in der Startelf entwickelt hat - und warum es nicht so gut ist, sich in die WM zu locken.
}%
}\\
\newline

\noindent
BERT (multilingual):\\

\noindent\fbox{%
\parbox{\textwidth}{%
Die DFB-Elf verliert das Halbfinale gegen Brasilien mit 7:1. Der Sieg ist nicht das erste Mal, dass die Südamerikaner den Sieg errungen haben - und die DFB-Frauen sind nicht zufrieden mit dem Spiel.
}%
}\\
\newline

\noindent
XLM-R (monolingual):\\

\noindent\fbox{%
\parbox{\textwidth}{%
Bundestrainer Joachim Löw gewinnt das Halbfinale gegen Gastgeber Brasilien mit 7:1. Die DFB-Elf holt sich den ersten Sieg in Serie, die deutsche Mannschaft holt den ersten Saisonsieg in Serie.
}%
}\\
\newline

\noindent
XLM-R (multilingual):\\

\noindent\fbox{%
\parbox{\textwidth}{%
Der DFB-Elf gewinnt gegen Brasilien und holt sich den Sieg gegen Brasilien. Der FC Bayern beim 7:1-Erfolg gegen Gastgeber Brasilien. Die Bayern beim 0:1 in der 2. Liga in der Einzelkritik.
}%
}\\
\newline

\noindent
BART (multilingual):\\

\noindent\fbox{%
\parbox{\textwidth}{%
Joachim Löw beat Brasilien 7:1 im Halbfinale gegen Gastgeber Brazil 7-1 in Berlin. Khedira meldete sich with Wadenproblemen kurzfristig ab, für ihn rückte Kramer in die Startformation.
}%
}\\
\newpage

\noindent
\textbf{Text 3: Definition von Inflation}\\

\noindent\fbox{%
\parbox{\textwidth}{%
Die Inflation ist ein anhaltender Prozess der Geldentwertung, der sich durch allgemeine Preiserhöhungen bemerkbar macht. Mit einer Geldeinheit kann dann ständig weniger gekauft werden, d.h. die Kaufkraft des Geldes vermindert sich dauernd. Nicht als Inflation gelten einmalige, vorübergehende, durch ungewöhnliche Vorkommnisse verursachte Preisniveauerhöhungen sowie Preissteigerungen für bestimmte Güter oder Produktionsfaktoren. Die Inflation wird gemessen am Anstieg eines das allgemeine Preisniveau am besten widerspiegelnden Preisindexes wie z.B. des Verbraucherpreisindexes für Deutschland. Der prozentuale Anstieg des Preisindexes in einem bestimmten Zeitraum wird als Inflationsrate bezeichnet. Beim Entstehen einer Inflation spielt besonders die Geldmenge in der Volkswirtschaft eine große Rolle. Steht der gesamtwirtschaftlichen Gütermenge eine zu große Geldmenge gegenüber, ist eine Bedingung für die Inflation gegeben. Übersteigt die gesamtwirtschaftliche Güternachfrage das gesamtwirtschaftliche Güterangebot, das kurzfristig nicht erhöht werden kann, sind steigende Preise die Folge und die Inflation setzt ein. Die Preissteigerungen lösen steigende Löhne aus, wegen des höheren Einkommens steigt die Nachfrage nach Gütern an. Die höheren Löhne bewirken jedoch auch steigende Kosten der Unternehmen, was wiederum zu Preissteigerungen für Güter führt.
}%
}\\
\newline

\noindent
\textbf{Zusammenfassungen von Text 3}\\

\noindent
Gold-Standard:\\

\noindent\fbox{%
\parbox{\textwidth}{%
Inflation ist ein anhaltender Prozess, bei dem die Kaufkraft des Geldes dauerhaft verringert wird. Sie wird durch die Inflationsrate gemessen. Die Preise steigen, die Löhne steigen, und damit steigt auch die Nachfrage. Höhere Löhne verursachen wiederum höhere Kosten für Unternehmen, weshalb die Preise für Güter weiter steigen.
}%
}\\
\newpage

\noindent
BERT (monolingual):\\

\noindent\fbox{%
\parbox{\textwidth}{%
Die Inflation ist ein anhaltender Prozess der Geldentwertung, der sich durch allgemeine Preiserhöhungen bemerkbar macht. Doch es gibt auch eine Lösung: Die Kaufkraft des Geldes verringert sich, die Kaufpreise steigen.
}%
}\\
\newline

\noindent
BERT (multilingual):\\

\noindent\fbox{%
\parbox{\textwidth}{%
Die Inflation ist ein anhaltender Prozess der Geldentwertung, der sich durch allgemeine Preiserhöhungen bemerkbar macht. Doch die Kaufkraft des Geldes verringert sich, weil die Kaufsumme immer höher ist als gedacht.
}%
}\\
\newline

\noindent
XLM-R (monolingual):\\

\noindent\fbox{%
\parbox{\textwidth}{%
Die Inflation ist ein anhaltender Prozess der Geldentwertung, der sich durch allgemeine Preiserhöhungen bemerkbar macht. Der Prozess wird von der Inflation erfasst und wird von einer Geldeinheit geführt. Die Preise für Güter und Produktionsfaktoren werden von der gegenwärtigen Inflation abgeleitet.
}%
}\\
\newline

\noindent
XLM-R (multilingual):\\

\noindent\fbox{%
\parbox{\textwidth}{%
Die Inflation ist ein anhaltender Prozess der Geldentwertung, der sich durch allgemeine Preiserhöhungen bemerkbar macht. Die Preise für bestimmte Güter werden immer teurer. Die Inflation ist ein Anlaufmodell für die steigenden Preise.
}%
}\\
\newline

\noindent
BART (multilingual):\\

\noindent\fbox{%
\parbox{\textwidth}{%
Die Inflation ist ein anhaltender Prozess der Geldentwertung, der sich durch allgemeine Preiserhöhungen bemerkbar macht. Mit einer Geldeinheit kann dann ständig weniger gekauft werden, d.h. die Kaufkraft des Geldes vermindert sich.
}%
}\\
\newpage

\noindent
\textbf{Text 4: Rechtsurteil zu Stornierungen während einer Pandemie}\\

\noindent\fbox{%
\parbox{\textwidth}{%
Storniert ein Reisender wegen einer Virus-Pandemie die Reise, so steht dem Reiseveranstalter kein Anspruch auf Entschädigung gemäß § 651 h Abs. 1 Satz 3 BGB zu, wenn er ebenfalls die Reise absagt. Zudem begründet die fehlende Therapiemöglichkeit und Impfung ein kostenloses Reise­rücktritts­recht. Dies hat das Amtsgericht Stuttgart entschieden. Dem Fall lag folgender Sachverhalt zugrunde: Wenige Tage vor Beginn einer mehrtägigen Flug- und Busrundreise durch Portugal im März 2020 erklärte die Reisende die Stornierung. Sie begründete dies mit dem sich ausbreitenden Corona-Virus. Die Reiseveranstalterin sagte zwei Tage später ebenfalls die Reise ab. Dennoch verlangte sie von der Reisenden die Zahlung einer Stornogebühr. Da sich die Reisende weigerte dem nachzukommen, kam es zu einem Klageverfahren. Das Amtsgericht Stuttgart entschied zu Gunsten der Reisenden. Der Reiseveranstalterin stehe kein Anspruch auf Entschädigung gemäß § 651 h Abs. 1 Satz 3 BGB zu. Denn zum einen könne sich die Reisende auf außergewöhnliche Umstände im Sinne von § 651 h Abs. 3 BGB berufen. Zum anderen sei zu berücksichtigen, dass die Reiseveranstalterin ebenfalls die Reise abgesagt hat.
}%
}\\
\newline

\noindent
\textbf{Zusammenfassungen von Text 4}\\

\noindent
Gold-Standard:\\

\noindent\fbox{%
\parbox{\textwidth}{%
Wenn ein Reisender die Reise aufgrund einer Viruspandemie storniert, hat der Reiseveranstalter keinen Anspruch auf Entschädigung. Reisende müssen daher keine Stornogebühren zahlen, wenn auch der Reiseveranstalter die Reise storniert hat.
}%
}\\
\newpage

\noindent
BERT (monolingual):\\

\noindent\fbox{%
\parbox{\textwidth}{%
Ein Reiseveranstalter muss die Reise absagen, wenn er die Reise nicht absagt. Das entschied das Bundesamt für Verkehrswesen. Das Urteil ist nicht nur für Reisende, sondern auch für das Reiserücktrittsrecht.
}%
}\\
\newline

\noindent
BERT (multilingual):\\

\noindent\fbox{%
\parbox{\textwidth}{%
Ein Reisender wegen einer Virus-Pandemie muss sich deswegen nicht entschädigen. Das Amtsgericht Stuttgart hat entschieden, dass die fehlende Therapiemöglichkeit und Impfung ein kostenloses Reiserücktrittsrecht geltend machen.
}%
}\\
\newline

\noindent
XLM-R (monolingual):\\

\noindent\fbox{%
\parbox{\textwidth}{%
Das Amtsgericht Stuttgart hat entschieden, dass ein Reiseveranstalter die Reise absagen muss, wenn er wegen einer Virus-Pandemie die Reise abgesagt hat. Das hat das Amt nun entschieden. Das Urteil könnte auch für Reisende gelten. Ein Überblick über die wichtigsten Punkte.
}%
}\\
\newline

\noindent
XLM-R (multilingual):\\

\noindent\fbox{%
\parbox{\textwidth}{%
Das Amtsgericht Stuttgart hat entschieden, dass ein Reiseveranstalter wegen einer Virus-Pandemie die Reise absagen muss, wenn er sich nicht auf Entschädigungen einsetzt.
}%
}\\
\newline

\noindent
BART (multilingual):\\

\noindent\fbox{%
\parbox{\textwidth}{%
Storniert ein Reisender wegen einer Virus-Pandemie die Reise, so steht dem Reiseveranstalter kein Anspruch auf Entschädigung gemäß § 651 h Abs. 1 Satz 3 BGB zu. Zudem begründet die fehlende Therapiemöglichkeit and Impfung ein kostenloses Reise­rücktritts­recht.
}%
}\\
\newpage

\noindent
\textbf{Text 5: Anleitung zum Händewaschen}\\

\noindent\fbox{%
\parbox{\textwidth}{%
Händewaschen ist eine der effektivsten Möglichkeiten, die Verbreitung von Keimen zu verhindern. Saubere Hände können die Ausbreitung von Keimen zwischen Personen und innerhalb einer ganzen Gemeinschaft verhindern. Befolgen Sie stets diese fünf Schritte. Befeuchten Sie Ihre Hände mit sauberem, fließendem Wasser (warm oder kalt), drehen Sie den Wasserhahn zu und verwenden Sie Seife. Schäumen Sie Ihre Hände ein, indem Sie sie mit der Seife aneinander reiben. Schäumen Sie die Handrücken, die Finger und die Fingernägel ein. Schrubben Sie Ihre Hände mindestens 20 Sekunden lang. Spülen Sie Ihre Hände gut unter fließendem, sauberem Wasser ab. Trocknen Sie Ihre Hände mit einem sauberen Handtuch ab oder lassen Sie sie an der Luft trocknen.
}%
}\\
\newline

\noindent
\textbf{Zusammenfassungen von Text 5}\\

\noindent
Gold-Standard:\\

\noindent\fbox{%
\parbox{\textwidth}{%
Händewaschen ist die effektivste Methode, um die Verbreitung von Keimen zu verhindern. Dazu sollte man die Hände mit kaltem Wasser waschen, dann Seife hinzufügen, 20 Sekunden lang ordentlich reiben und anschließend unter kaltem Wasser abspülen.
}%
}\\
\newpage

\noindent
BERT (monolingual):\\

\noindent\fbox{%
\parbox{\textwidth}{%
Saubere Hände können die Verbreitung von Keimen verhindern. Aber wie kann das sein? Ein paar Tipps, wie Sie die Verbreitung der Keimen vermeiden. Ein Überblick.
}%
}\\
\newline

\noindent
BERT (multilingual):\\

\noindent\fbox{%
\parbox{\textwidth}{%
Keine Keime, keine Keime: Kein Wunder, dass Keime die Ausbreitung von Keimen zwischen Menschen und innerhalb einer ganzen Gemeinschaft verhindern. Wie man sich damit beschäftigen kann und wie man den Weg zu Keimen schützt.
}%
}\\
\newline

\noindent
XLM-R (monolingual):\\

\noindent\fbox{%
\parbox{\textwidth}{%
Händewaschen ist eine der effektivsten Möglichkeiten, die Verbreitung von Keimen zu verhindern. Dabei ist es eine effektive Methode, um die Verbündeten zu bewahren und die Ausbrüche zu vermeiden. Ein paar einfache Schritte.
}%
}\\
\newline

\noindent
XLM-R (multilingual):\\

\noindent\fbox{%
\parbox{\textwidth}{%
Handwaschen ist eine der effektivsten Möglichkeiten, die Verbreitung von Keimen zu verhindern. Ein Überblick über die effektivsten Mittel. Zumindest, wenn die Händewaschen zu sauber sind und die Hände mit sauberem Wasser befüllt sind. Die wichtigsten Tipps zum Umgang mit Keimen.
}%
}\\
\newline

\noindent
BART (multilingual):\\

\noindent\fbox{%
\parbox{\textwidth}{%
Sauberes Händewaschen kann die Ausbreitung von Keimen zwischen Personen verhindern. Befolgen Sie stets diese fünf Schritte.
}%
}\\
