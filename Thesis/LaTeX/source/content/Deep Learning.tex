\chapter{Deep Learning}
\thispagestyle{fancy}
\label{chap:Deep Learning}

Notizen:
\begin{itemize}
	\item Deep Learning definieren
	\item Machine Learning erwähnen
	\item Siehe Abstract im Exposé
\end{itemize}


\section{Neuronale Netze}
Notizen:
\begin{itemize}
	\item Neuronale Netze definieren
	\item Historie beschreiben
	\item Funktionsweise und ausgewählte Komponenten beschreiben
\end{itemize}


\section{Reinforcement Learning}
Notizen:
\begin{itemize}
	\item Reinforcement Learning definieren. auch Deep Reinforcement Learning als Kombination aus neuronalen Netzen und Reinforcement Learning, beides Unterkapitel des Deep Learning selbst, gute Zusammenfassung zu Beginn des einen Abschnittes hier: \url{https://medium.com/analytics-vidhya/deep-reinforcement-learning-deeprl-for-abstractive-text-summarization-made-easy-tutorial-9-c6914999c76c}
	\item Bisherige Errungenschaften und Eigenschaften erwähnen
	\item Funktionsweise und ausgewählte Komponenten ggf. in Unterkapiteln beschreiben
	\item \url{https://www.learndatasci.com/tutorials/reinforcement-q-learning-scratch-python-openai-gym/#}
\end{itemize}


\section{Architekturen}
Notizen:
\begin{itemize}
	\item Existenz und Notwendigkeit verschiedener Architekturen ankündigen, ggf. in spätere Kapitel verlegen, bspw. zum abstraktiven Ansatz
	\item Später benötigte Architekturen hier beschreiben
	\item Diversität der existierenden Architekturen (wie im Forschungsstand bereits erwähnt) hervorheben
	\item "Reinforcement Learning comes to the rescue" aus \url{https://towardsdatascience.com/deep-learning-models-for-automatic-summarization-4c2b89f2a9ea} einbinden
	\item Encoder/ Decoder, Self-Attention, Seq to Seq, Transformer Model (Recherche + Vergleich)
\end{itemize}


\subsection{MLP}
Notizen:


\subsection{RNN}
Notizen:


\subsection{LSTM}
Notizen:


\subsection{DQN}
Notizen:


\section{Hyperparameter}
Notizen:
\begin{itemize}
	\item Hyperparameter vorstellen
	\item Notwendigkeit und Einfluss von Hyperparametern beschreiben
\end{itemize}
