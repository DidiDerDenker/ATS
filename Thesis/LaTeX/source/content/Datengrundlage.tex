\chapter{Datengrundlage}
\thispagestyle{fancy}
\label{chap:Datengrundlage}

Notizen:
\begin{itemize}
	\item Modelle erfordern keine gelabelten Daten, wohl aber gesichtete Daten
	\item Datengrundlage besteht aus frei verfügbaren allgemeinsprachlichen, ausreichend langen und deutschsprachigen Daten, verschiedene Herkünfte
	\item Skripte zur Datenakquise in Python, damit Wikipedia und ähnliche Texte gezogen, Skript beschreiben, Zielform der Daten beschreiben, rekursiv Wikipedia durchforsten
	\item Später dann zwecks Adaption auch unternehmensinterne fachspezifische Daten möglich, genauer beschreiben, perspektivisch (ggf. im Ausblick erst erwähnen) fachspezifische, dialogorientierte oder auch mehrsprachige Modelle möglich, dementsprechend mehr Daten benötigt, Mindestlänge definieren und begründen, es wird angenommen, dass 1000 Wörtern vorliegen müssen, um eine Zusammenfassung erforderlich zu machen, später auch Tests auf 1000 Wörtern bspw. denkbar
	\item Beschreiben, auf welche Form die Daten gebracht werden, ggf. auf Python-Skripte verweisen
	\item Weitergehende Besonderheiten innerhalb der Texte werden toleriert, da diese auch im Praxisbetrieb auftreten könnten und somit gekannt werden sollten
	\item Siehe Abstract im Exposé
\end{itemize}
