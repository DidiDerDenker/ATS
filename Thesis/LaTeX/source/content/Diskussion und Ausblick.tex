\chapter{Diskussion und Ausblick}
\thispagestyle{fancy}
\label{chap:Diskussion und Ausblick}

Notizen:
\begin{itemize}
	\item Adaptive Learning für die Modelle ansatzweise vorstellen
	\item Modelle für mehrere Sprachen trainieren
	\item Modell auf Dialogcharakter adaptieren, um es in der Verdichtung von Protokollen einer Videosprechstunde zu nutzen, bzw. generell bspw. Meetings zusammenzufassen
	\item Forschungsstand und SOTA-Modelle hierfür beschreiben (vgl. Paper: „Abstractive Dialogue Summarization with Sentence-Gated Modeling Optimized by Dialogue Acts“ und “Using a KG-Copy Network for Non-Goal Oriented Dialogues”), bereits Architekturen vorstellen (vgl. „Automatic Dialogue Summary Generation of Customer Service“ und „Dialogue Response Generation using Neural Networks and Background Knowledge“ und „Global Summarization of Medical Dialogue by Exploiting Local Structures”)
	\item Gelb markierte Literatur sichten und verwenden, Datumsangaben aktualisieren
	\item Siehe Abstract im Exposé
\end{itemize}
