\chapter{Sprachtechnische Adaption}
\thispagestyle{fancy}
\label{chap:Sprachtechnische Adaption}

\noindent
Unter Kenntnis der Architektur des abstraktiven Ansatzes und der Baseline des englischsprachigen Modells wird nun ergründet, wie eine Adaption auf die deutsche Sprache erfolgen kann. Dies wird mithilfe verschiedener Experimente erprobt, welche sich weiterhin auf die bekannten Metriken stützen.


\section{Architektur}
\noindent
Die bislang genutzte Architektur, welche bekanntermaßen Transformer integriert, kann auf Grundlage umfangreicher verschiedensprachiger ungelabelter Textdaten im Sinne des \ac{DL} und \ac{TL} multilingual vortrainiert werden, ohne architektonische Anpassungen vornehmen zu müssen. Hierbei werden sprachübergreifende verborgene Strukturen erlernt, um anschließend monolingual davon zu profitieren. Zudem wird dem Problem entgegengewirkt, dass sprachintern zu wenig Textdaten zur Verfügung stehen \cite{MOB20}. Modelle, welche organisationsextern bereits multilingual vortrainiert und bereitgestellt wurden, sind beispielsweise die multilinguale Version von \ac{BERT} und die \ac{XLM-R}. Hinsichtlich einer anschließenden deutschsprachigen Nutzung in der \ac{ATS} ist weiterhin ein entsprechendes sprachbezogenes Training erforderlich. Dies wird im Verlauf der Experimente hinreichend abgehandelt.


\section{Experimente}
\noindent
In einem initialen Experiment wird die multilinguale Version von \ac{BERT} mit einem Mix aus allen verfügbaren deutschsprachigen Daten trainiert. Hierbei entstehen die folgenden \ac{ROUGE}-Scores: R-Recall: 24.56, R-Precision: 15.05, R-Measure: 17.25. Dies überliegt dem \ac{SOTA} zwar deutlich, ist allerdings auf die Struktur der Trainingsdaten zurückzuführen, da mehr als die Hälfte aller Daten aus Wikipedia-Artikeln stammen.
\newpage

\noindent
Dennoch sei nachfolgend die Zusammenfassung des in Anhang B einzusehenden deutschsprachigen Textes gezeigt. Dieser gleicht strukturell sowie inhaltlich dem bereits beschriebenen Text in Anhang A. In der Folge schließen sich verschiedene jeweils aus dem vorherigen Schritt abgeleitete Experimente an.\\

\noindent\fbox{%
\parbox{\textwidth}{%
In New York ist ein Flugzeug aus dem World Trade Center auf dem Nordturm des World Trade Centers gebrannt. In den USA ist es zu einem Unfall gekommen. Nun ist das Land wieder in Schock, wo es sich um einen Brand ereignete. Es war der erste Unfall, der sich in den USA ereignet hat.
}%
}\\

\noindent
Die inhaltliche Schwäche ist für den menschlichen Leser unschwer erkennbar auch ohne Kenntnis über den Originaltext. Daher wird nun die multilinguale Version von \ac{BERT} durch eine eigens für die deutsche Sprache vortrainierte Version ausgetauscht. An der deutschsprachigen Datengrundlage wird zunächst nichts verändert. Hierbei entstehen die folgenden \ac{ROUGE}-Scores: R-Recall: 24.20, R-Precision: 14.65, R-Measure: 16.85. Dies unterliegt den \ac{ROUGE}-Scores der multilingualen Version von \ac{BERT}. Letzterer scheint tatsächlich von den verborgenen Strukturen anderer Sprachen zu profitieren.\\

\noindent
Mit dem Wissen, dass multilingual vortrainierte Modelle sogar die Qualität monolingualer Modelle im deutschsprachigen Raum übersteigen, wird nun \ac{BERT} durch \ac{XLM-R} ersetzt, um sich geeigneten Ergebnissen zu nähern. Der Tokenizer wird ebenfalls entsprechend ausgewechselt. Es entstehen jedoch wider Erwarten die folgenden \ac{ROUGE}-Scores: R-Recall: 23.02, R-Precision: 13.95, R-Measure: 16.09. Mutmaßlich sind die Ergebnisse von \ac{XLM-R} doch nicht besser als die von \ac{BERT}. Dies zeigt auch die nachfolgende exemplarische Zusammenfassung von Anhang B, welche subjektiv als inhaltlich mangelhaft einzustufen ist.\\

\noindent\fbox{%
\parbox{\textwidth}{%
Der 11. Flugtag der Weltturm-Weltmeisterschaft 2001 fand am 11. September 2001 in New York City statt und war das erste Mal in der Geschichte des World Trade Centers. Das National September 11 Memorial and Museum in Manhattan ist ein historischer Gedenkpavillon in Manhattan. Es befindet sich in der Nähe des World Trade Centers in Manhattan.
}%
}\\

% TODO: BART probieren, mit bestem Modell fortfahren, diese Entscheidung begründen, dann weiterhin geeigneten Ergebnissen nähern, Korpora variieren, Data Augmentation etc.

% TODO: Modelle, die experimentell namentlich in der Thesis erwähnt werden, im theoretischen Kapitel ergänzen, hier mit einem weiteren Block mit ROUGE-Scores und Beispielzusammenfassung hinzufügen

% TODO: Nachfolgenden Abschnitt bzgl. Entfernen der ersten Wikipedia-Absätze mit korrektem Modellnamen und ROUGE-Scores aktualisieren, Änderungen kritisch kommentieren, damit begründet weitere Anpassungen vornehmen

% TODO: Sliding-Window-Approach bei zu großen Texten beschreiben und entwickeln, beim Laden der Texte, aber als separate Methode, die auf einzelne Texte anwendbar ist, trotzdem mit Map als Batch-Verarbeitung, über Bool in der Config beim Training auswählbar machen, im Beispiel als Methode einbinden

% TODO: Wikipedia-Anteil reduzieren, Modell nur auf Wikipedia trainieren, nur auf ZEIT evaluieren, dann nur auf ZEIT trainieren und evaluieren

% TODO: Kapitel besser strukturieren (format-technisch, inhaltlich), Beschreibung der Datengrundlage nochmal überprüfen, Lorem ipsum und TODO's in beiden Kapiteln herausarbeiten, Hyperparameteroptimierung auf 10-20 Prozent der Daten vornehmen, Verdichtung der Zusammenfassung verdeutlichen, d.h. Token-Reduktion, zudem die prozentuale Kompressionsrate angeben (75% von 512 auf 128 Token), d.h. mit technischen Anpassungen können auch 5 DIN A4-Seiten um x Prozent verdichtet werden (Wie lang ist der Eingangstext? Wie lang ist der Ausgangstext? Wie geht das Modell mit längeren Texten um?)

% TODO: \cite{YAN19} S. 4 rechts, Herausforderung: Encoder overfitted, Decoder underfitted oder andersherum, wird durch HuggingFace-Framework vorgebeugt, \cite{YAN19} S. 5 oben für Evaluation, Vergleichstabelle der Experimente einbinden und beschreiben, typisches Diagramm zur Visualisierung des Trainingsprozesses anfügen, Verhalten des Modells interpretieren und Anpassungen ableiten, bspw. Exploitation wegen der Struktur der Artikel nochmal aufgreifen, ggf. erst bei der sprachtechnischen Adaption, erwähnen, dass dies als Experiment genügt, sprachtechnische Anpassungen dann erst im nächsten Kapitel, Referenzzusammenfassungen mit ROUGE und BLEU bewerten, um Vergleichswerte nennen zu können, Texte manuell zusammenfassen, um ebenfalls einen Vergleichswert von ROUGE und BLEU zu haben\\
