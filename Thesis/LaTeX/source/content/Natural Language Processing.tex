\chapter{Natural Language Processing}
\thispagestyle{fancy}
\label{chap:Natural Language Processing}

Notizen:
\begin{itemize}
	\item Natural Language Processing definieren, e.g. Natural Language Understanding?
	\item NLP ist Optimierungslösung, d.h. es gibt keine eindeutige und damit im mathematischen Sinne analytische Lösung, Beispiel bei der Textzusammenfassung: Selbst Menschen können Texte auf verschiedene Arten und Weisen zusammenfassen, und verschiedene Varianten können korrekt sein
	\item NLU ist Teilgebiet des NLP	
	\item Umfang der Anwendungsgebiete andeuten
	\item Natural Language Generation bspw. zum Generieren von Texten anhand von Stichworten benutzen, sollte bereits in gutem Zustand implementierfähig sein, möglicherweise Strukturen hiervon für die Generierung der Zusammenfassung verwenden, NLP-Links: \url{https://www.analyticsvidhya.com/blog/2020/08/build-a-natural-language-generation-nlg-system-using-pytorch/}, \url{https://www.analyticsvidhya.com/blog/2019/09/introduction-to-pytorch-from-scratch/?utm_source=blog&utm_medium=Natural_Language_Generation_System_using_PyTorch}, \url{https://courses.analyticsvidhya.com/courses/natural-language-processing-nlp?utm_source=blog&utm_medium=Natural_Language_Generation_System_using_PyTorch}
	\item \url{https://github.com/adbar/German-NLP#Data-acquisition}
	\item \url{https://github.com/JayeetaP/mlcourse_open/tree/master/jupyter_english}
	\item Spacy: \url{https://spacy.io/usage/processing-pipelines#pipelines}
	\item Lemmatizer: \url{https://github.com/Liebeck/spacy-iwnlp}
	\item Transfer Learning with German BERT? \url{https://deepset.ai/german-bert} -> Modell muss die deutsche Sprache nicht alleine und von neu  mit den Trainingsdaten lernen, sondern erhält einen großen Vorsprung, BERT ist Modell, welches der Transformer-Architektur nachkommt, d.h. Transformer sind bestimmte Architekturen, eventuell hiermit die Struktur dieses Kapitels überarbeiten, hier für vor allem aus meinem privaten Verzeichnis das Paper "Pre-Training of Deep Bidirectional Transformers for Language Understanding using BERT" nutzen
	\item GLoVe-Embeddings nutzen, weil TF-IDF etc. nicht den Kontext eines Satzes betrachten
	\item Supervised Learning nutzen, aber es ist eventuell nicht genug, hier kommt bspw. Transfer Learning mit BERT zur Abhilfe, zudem bspw. semi-supervised Learning mit Auto-Encoders? Self-supervised Training
	\item Siehe Abstract im Exposé
\end{itemize}


\section{Vorverarbeitung}
Notizen:
\begin{itemize}
	\item Pipeline der Vorverarbeitung als Voraussetzung hervorstellen
	\item Relevanz von Capitalization, Punctuation, Zeilenumbrüchen klären, auch im Negativfall begründen und belegen, Satzzeichen für die Minimierung von Zwei- oder Uneindeutigkeiten berücksichtigen
\end{itemize}


\subsection{Textbereinigung}
Notizen:
\begin{itemize}
	\item Siehe vergangene Aufgaben in GitHub
\end{itemize}


\subsection{Tokenisierung}
Notizen:


\subsection{POS-Tagging}
Notizen:


\subsection{Lemmatisierung}
Notizen:
\begin{itemize}
	\item Lemmatisierung eventuell irrelevant, weil Wort-Tokenisierung bei modernen Architekturen und Modellen oftmals ausreicht
	\item Nach erfolgreichem Aufsetzen der Pipeline kann man die Eingangsdaten testweise immer noch der Lemmatisierung oder weiteren Vorverarbeitungsschritten unterziehen, um deren Auswirkungen zu messen
\end{itemize}


\subsection{Entfernen von Stoppwörtern}
Notizen:


\section{Merkmalsextraktion}
Notizen:
\begin{itemize}
	\item Relevanz für extraktiven Ansatz beschreiben (vgl. Paper: „Automatic Text Summarization“)
	\item Relevanz für abstraktiven Ansatz, falls vorhanden, beschreiben
	\item Metriken selbst weiterentwickeln und ausreifen
	\item Siehe: \url{https://scikit-learn.org/stable/modules/feature_extraction.html}
\end{itemize}


\subsection{Übereinstimmung mit dem Titel}
Notizen:


\subsection{Satzposition}
Notizen:


\subsection{Satzähnlichkeit}
Notizen:


\subsection{Satzlänge}
Notizen:


\subsection{Domänenspezifische Wörter}
Notizen:


\subsection{Eigennamen}
Notizen:


\subsection{Numerische Daten}
Notizen:
