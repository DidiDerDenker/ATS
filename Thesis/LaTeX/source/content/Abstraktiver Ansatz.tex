\chapter{Abstraktiver Ansatz}
\thispagestyle{fancy}
\label{chap:Abstraktiver Ansatz}

\noindent
Unter Kenntnis der Grundlagen des \ac{DL} und des \ac{NLP} wird nun eine Architektur konzipiert und beschrieben, welche die \ac{ATS} gemäß des abstraktiven Ansatzes ermöglicht. Hierfür werden verschiedene Experimente durchgeführt, deren Training stets auf der beschriebenen Datengrundlage erfolgt.\\


\section{Metriken}
\noindent
Zuvor sind ausgewählte Metriken offenzulegen, mit denen die Qualität der \ac{ATS} gemessen werden kann: \ac{ROUGE} und \ac{BLEU}. In der Wissenschaft werden \ac{ATS}-Modelle meist mithilfe des \ac{ROUGE}-Scores evaluiert und verglichen. Dabei erfordern diese Metriken eine \ac{REFZ} zu jeder \ac{SYSZ}, welche maschinell generiert wurde. Dies ist unabhängig davon, ob das Training überwacht oder unüberwacht durchgeführt wurde. Allgemein unterliegen diese Metriken der Herausforderung, dass es für einen gegebenen Text keine objektiv beste Zusammenfassung gibt. Folglich können verschiedene \ac{SYSZ} oder sogar \ac{REFZ} gleich gut sein. Dies ist statistisch sehr schwer zu bewerten, zumal selbst Menschen aufgrund ihrer subjektiven Bewertungsweise nicht einheitlich definieren können, welche Faktoren für eine gute Zusammenfassung stehen. Weiterhin werden in den Metriken menschliche Bewertungsfaktoren wie beispielsweise Lesbarkeit nicht berücksichtigt \cite{LEM20}.\\


\subsection{ROUGE}
\noindent
\ac{ROUGE} kann zunächst mithilfe folgender Kennzahlen weitergehend differenziert werden: Recall, Precision und Measure. Dabei quantifiziert der Recall-Score den Anteil der sowohl in \ac{REFZ} als auch in \ac{SYSZ} vorkommender Wörter gemäß folgender Formel:\\
$$\frac{\text{Anzahl übereinstimmender Wörter}}{\text{Anzahl der Wörter in der REFZ}}$$
\newpage

\noindent
Sei hierfür verkürzt \glqq Der Sommer war sehr warm\grqq{} als \ac{REFZ} und \glqq Der Sommer war wieder sehr warm\grqq{} als \ac{SYSZ} gegeben. Dann gilt: $\text{Recall} = \frac{5}{5} = 1.0$. Trotzdem sollen \ac{SYSZ} nicht unendlich lang werden, um alle Wörter der \ac{REFZ} abzudecken, sondern weiterhin den eigentlichen Sinn einer Zusammenfassung erfüllen. Hier quantifiziert der Precision-Score den Anteil der tatsächlich relevanten Wörter gemäß folgender Formel:\\
$$\frac{\text{Anzahl übereinstimmender Wörter}}{\text{Anzahl der Wörter in der SYSZ}}$$ \newline

\noindent
Wie man sieht, ändert sich nur der Nenner. Im oben genannten Beispiel gilt somit: $\text{Precision} = \frac{5}{6} = 0.8\overline{3}$. Nimmt man ferner \glqq Der letzte Sommer war wieder sehr warm und trocken\grqq{} als neue \ac{SYSZ} an, dann reduziert sich der Precision-Score aufgrund der erhöhten Anzahl unrelevanter Wörter wie folgt: $\text{Precision} = \frac{5}{9} = 0.5\overline{5}$. Weiterhin kann der Measure-Score als gewöhnlicher F-Score verstanden und interpretiert werden. Er ergibt sich als harmonisches Mittel zwischen dem Recall und der Precision, womit er beide Scores berücksichtigt \cite[S.~1-3]{LIN04}.\\

\noindent
\ac{ROUGE} wird allgemein auch als \ac{ROUGE}-N geschrieben, wobei das N bestimmt, ob obige Kennzahlen auf Grundlagen von Uni-, Bi- oder Trigrammen berechnet werden sollen. Im genannten Beispiel wurden also die \ac{ROUGE}-1 Recall-, Precision- und Measure-Scores berechnet. Zudem existieren Ansätze, welche die \ac{LCS} verfolgen. Diese werden hier jedoch vernachlässigt.\\

\noindent
Trotz oder gerade wegen der wissenschaftlichen Verbreitung des \ac{ROUGE}-Scores kommt immer mehr Kritik auf. Demnach kann \ac{ROUGE} beispielsweise nicht zwischen verschiedenen aber bedeutungsähnlichen Wörtern unterscheiden. Dies führt tendenziell zu einer schlechteren Bewertung, obgleich ein gegebener Text etwa in einer entsprechenden \ac{SYSZ} präzise zusammengefasst wurde. Außerdem wird den Texten zur Berechnung des \ac{ROUGE}-Scores Kleinschreibung abverlangt, unabhängig von den vorgeschalteten Modellen. Die Bewertung geschieht also eher auf syntaktischer als auf semantischer Basis. Aufgrund der bereits genannten weitreichenden Nutzung des \ac{ROUGE}-Scores und der damit gegebenen Vergleichbarkeit kommt er trotzdem in dieser Arbeit zum Einsatz \cite[S.~5]{LIN04}.
\newpage

	
\subsection{BLEU}
\noindent
\ac{BLEU} kommt der Funktionsweise von \ac{ROUGE} weitestgehend gleich. Demnach repräsentiert der Score ebenfalls die Ähnlichkeit längendefinierter N-Gramme. Dabei wird weiterhin für jede \ac{SYSZ} eine entsprechende \ac{REFZ} gefordert. Beide Metriken funktionieren indes sprachunabhängig. Im Unterschied zu \ac{ROUGE} führt \ac{BLEU} einen multiplikativen Bestrafungsterm ein, um zu kurze \ac{SYSZ} zu entwerten. Dies ist nicht notwendig, wenn die \ac{SYSZ} länger ist als die \ac{REFZ}. Der Precision-Score berücksichtigt dies bereits \cite[S.~5]{PAP02}.\\

\noindent
Obgleich der \ac{BLEU}-Score primär die Bewertung von Übersetzungen unterstützt, eignet er sich gewissermaßen auch für \ac{ATS}-Aufgaben. Zuletzt konnte wissenschaftlich bewiesen werden, dass der \ac{BLEU}-Score recht gut mit menschlichen Bewertungen korreliert. Trotz ähnelnden Nachteilen zu denen des \ac{ROUGE}-Score wird auch der \ac{BLEU}-Score vergleichend in dieser Arbeit verwendet \cite[S.~6-7]{PAP02}.


\section{Architektur}
\noindent
Einleitungstext von unten anpassen und übernehmen, mit Notizen darunter fortsetzen

\noindent
Hierfür werden mithilfe diverser Experimente entsprechende Modelle trainiert, welche allesamt mit den beschriebenen Daten arbeiten.
relevanter Architekturen, entsprechender \ac{NLP}-Grundlagen und der Datengrundlage kann nun eine gesamtheitliche Architektur für die \ac{ATS} konzipiert und trainiert werden. Dafür sind zunächst einige Vorbemerkungen zu spezifizieren.\\

\noindent
Die Textinhalte aller Korpora bedürfen keiner weitergehenden Vorverarbeitung im herkömmlichen Sinne. Diese ist bekanntermaßen sehr individuell und stark modellabhängig. Unter Verwendung der als sehr robust geltenden Transformer-Architekturen entfällt daher die sonst übliche Textbereinigung sowie die Textnormalisierung. Dies unterliegt der Annahme, dass Transformer-Architekturen potenziell aus jeder Eigenart ein relevantes Feature schaffen können, welches das spätere Ergebnis begünstigt. Von der zugeführten Interpunktion und den vielfältigen Wortformen wird sich indes erhofft, potenzielle Mehr- oder Uneindeutigkeiten zu minimieren. Das Fine-Tuning sollte darüber hinaus unter gleichen Bedingungen wie das initiale Training stattfinden. Gleichzeitig sinkt hierdurch der vorverarbeitende Aufwand und damit auch etwaige Wartezeiten bei der praktischen Anwendung bereits trainierter Modelle in Echtzeit. Dennoch ist es möglich, bestimmte Vorverarbeitungsschritte a posteriori zu implementieren. Die Auswirkungen auf das Modell und die entsprechenden Ergebnisse sind somit messbar.\\

... Architektur beschreiben, auf bereits beschriebene Inhalte eingehen, Notizen beachten, Tokenlänge o.ä. bemerken, Quellen!

TODO: Leveraging-Paper hier umfänglich beschreiben, auf dokumentierte Inhalte dieser Arbeit eingehen + \cite{VAS17}

Nitsche lesen und einarbeiten

Notizen:
\begin{itemize}
	\item Quelle: \cite{NIT19} ab Kapitel 4 (!)
	\item Abgrenzung zum extraktiven Ansatz beschreiben
	\item Vorteile gegenüber referenzierten Modellen herausstellen
	\item Generierung neuer Sätze sowohl mit vorkommenden als auch mit nicht-vorkommenden Wörtern (vgl. Paper: „Automatic Text Summarization with Machine Learning“)
	\item Extraktive Zusammenfassung: Potenzielle Features bzw. Kennzahlen: Übereinstimmung mit dem Titel, Satzposition, Satzähnlichkeit, Satzlänge, domänenspezifische Wörter, Eigennamen, numerische Daten
\end{itemize}

Notizen:

BERT als Encoder \& Decoder nutzen, Architekturen und TL dementsprechend aufgreifen (YAN19 auf S. 1, ROT20 auf S. 2 rechts und S. 6 unten)

Encoder zur NLU und Decoder zur NLG, d.h. BERT oder andere Transformer als vortrainiertes multilinguales Modell für Encoder/ Decoder nutzen

BERT ist außerdem austauschbar, durch sowohl größere als auch kleinere Modelle

Multilingualität architektonisch ergründen, Ausblick auf Adaption von EN->DE, erwähnen, dass diese Modelle die Encoder oder auch Decoder ersetzen können, hierzu populäre Ansätze wie \cite{ROT20}

Notebook \url{https://colab.research.google.com/drive/1WIk2bxglElfZewOHboPFNj8H44_VAyKE?usp=sharing#scrollTo=4M2uzGLV9a_O} im abstraktiven Ansatz einarbeiten und unter "Analysis" referenzieren, jeweils maximale Token-Länge oder verschieden trainierte Versionen hervorheben, ggf. LongFormer als Encoder benutzen

Notizen:
\begin{itemize}
	\item Eignung von Python für ML/ DL
	\item Anmerkungen zum CNN-Korpus aus der Datengrundlage aufgreifen und mit Colab-Notebook vervollständigen, d.h. wie viele Texte sind länger als BERT-MAX 512 Token? Was 	haben Experimente ergeben? Token-Länge/ Longformer? Gibt es alternative Lösungen? Z.B. Longformer -> Ausblick
	\item Multilingual BERT: \url{https://huggingface.co/bert-base-multilingual-cased}
	\item \url{https://colab.research.google.com/drive/1WIk2bxglElfZewOHboPFNj8H44_VAyKE?usp=sharing#scrollTo=ZwQIEhKOrJpl}, \url{https://huggingface.co/transformers/training.html#trainer}, \url{https://www.quora.com/What-is-the-difference-between-FP16-and-FP32-when-doing-deep-learning}, \url{https://margaretmz.medium.com/setting-up-aws-ec2-for-running-jupyter-notebook-on-gpu-c281231fad3f} + Jupyter Notebook auf Tesla 80K \url{https://www.nvidia.com/de-de/data-center/tesla-k80/}
	\item \cite{ROT20} um Encoder und Decoder zu ersetzen, laut beschriebener Encoder-Decoder-Architektur + Language Model + Transfer Learning in Kombination, also die drei Grundlagenkapitel werden hier zusammengeführt, sogenannter Warm-Start des Encoder-Decoder-Models, d.h. kein initiales Training erforderlich, welches im Bereich von NLP-Tasks ein Neuerlernen einer Sprache bedeuten würde, Architektur skizzieren, in der sowohl Encoder als auch Decoder "warm gestartet" werden, Fine-Tuning auch skizzieren, Warm-Starting beschreiben, wie hier in der Theorie beschrieben: \url{https://colab.research.google.com/drive/1WIk2bxglElfZewOHboPFNj8H44_VAyKE?usp=sharing#scrollTo=Gw3IZYrfKl4Z}, Skizzen von dort nutzen, Encoder-Decoder-Weight-Sharing beschreiben, zuvor muss hier erstmal beschrieben werden, wie die Encoder-Decoder-Architektur konkret aussieht, also BERT o.ä., dann verschiedene Skizzen und Dinge wie Weight-Sharing beschreiben, prinzipiell das Colab-Notebook nutzen, Multilingualität behandeln
	\item \cite{YAN19} S. 4 rechts, S. 5 oben für Evaluation, S. 6 links unten für Konfiguration
	\item Frage: Fine-Tuning auf CNN/ Dailymail für Funktionalität der ATS auf englischer Sprache, d.h. Modell soll auf ATS fokussiert werden (bspw. von HuggingFace übernehmen), dann weiteres Fine-Tuning auf deutschen Korpus zur sprachlichen Adaption, also zweistufig?
	\item Modellauswahl begründen, d.h. warum "Transformer"? Warum genau dieses vortrainierte Modell? Auf vorherige Inhalte der Masterarbeit verweisen
	\item Transformer-to-Transformer-Architektur beschreiben, Grundlagen des TL, der Encoder/ Decoder und der Embeddings/ Language Model Representations etc. aufgreifen, bspw. Bert2Bert, Longformer2Roberta wie in \url{https://huggingface.co/patrickvonplaten/longformer2roberta-cnn_dailymail-fp16/blob/main/README.md}
	\item Netzwerk des abstraktiven Ansatzes als Pipeline skizzieren
	\item Vorgehen bei der Datenverarbeitung/ -vorbereitung wie im Notebook beschreiben, d.h. warum cased/ uncased? Token-Länge von BERT o.ä. als Begründung erwähnen
	\item Transformers-Library -> Scores in Excel, funktioniert gut als Benchmark/ "Nullfall"
	\item Komponenten wie den Encoder oder den Decoder mit der behandelten Theorie auf die durch die Datengrundlage gestellten Anforderungen mappen, d.h. tatsächlich eine Auswahl aus den vorgestellten theoretischen Inhalten treffen und stets ausführlich begründen
	\item Seq2Seq-Trainer wie im Notebook beschreiben, Notebook nochmal durchlesen und wichtige Informationen sinnvoll einbinden
	\item Seq2Seq \url{https://github.com/yaserkl/RLSeq2Seq#dataset} -> Fehler
	\item Seq2Seq-Library: \url{https://github.com/dongjun-Lee/text-summarization-tensorflow} -> Done, aber Scores auf meinem Datensatz nicht evaluierbar, da Aufbau des Vocabularies und Training auf meinem Korpus ausstehend ist, aber zu rechenintensiv ist, alternativ nur Scores auf anderem Korpus auswertbar, Azure ML vs. AWS SageMaker?
	\item Deep Reinforcement Learning (DeepRL) for Abstractive Text Summarization \url{https://medium.com/analytics-vidhya/deep-reinforcement-learning-deeprl-for-abstractive-text-summarization-made-easy-tutorial-9-c6914999c76c} -> Rouge-Scores ausschließlich auf dem CNN-Korpus berechnet, Anpassungen an den Daten, an der Code-Architektur und an den Modellen möglich -> Zurückgestellt aber bei Bedarf mit Potenzial
	\item BERT-Encoder Transformer-Decoder: Paper \url{https://arxiv.org/pdf/2008.09676.pdf}, Code \url{https://github.com/nlpyang/PreSumm}, Results \url{https://paperswithcode.com/paper/abstractive-summarization-of-spoken#code} -> In Progress...
	\item Deep Reinforced Model with PyTorch: \url{https://github.com/rohithreddy024/Text-Summarizer-Pytorch} -> TBD
\end{itemize}


\section{Training}
Notizen:
\begin{itemize}
	\item ZIH-Cluster mit HPC/ Taurus aus IT- /ML-Infrastruktur zum Training, Spezifikationen etwas beschreiben
	\item Setup beschreiben \url{https://docs.aws.amazon.com/dlami/latest/devguide/tutorials.html}, \url{https://youtu.be/pK-LYoRwp-k?list=WL}
	\item Konfiguration
	\item Training verschiedener Modelle
	\item Kompressionsrate der Referenzzusammenfassungen in Bezug auf die Originaltexte liegt bei 12 Prozent, d.h. mit einer gewissen Toleranz wird die maximale Zusammenfassungslänge auf 15 Prozent des Originaltextes festgelegt
	\item Architektur beschreiben, d.h. IT-Infrastruktur + GPU-Architektur
\end{itemize}


\section{Evaluation}
Notizen:
\begin{itemize}
	\item Evaluation verschiedener Modelle mit geeigneter Vergleichstabelle
	\item Praktische Nutzung durch Implementation eines vortrainierten Modells in ein Skript oder eine Software
	\item Es muss eine Metrik existieren, mit der man die Genauigkeit bzw. Qualität der Zusammenfassung messen kann, d.h. man möchte die Texte nicht mit menschlich generierten Zusammenfassungen vergleichen, sondern automatisiert lernen, ggf. sollte man auch Grammatik und Inhalt separat prüfen
	
	\item Bei der Anwendung einer Architektur, in der das Modell durch Reinforcement Learning trainiert wird, braucht man keine massenhaft menschlich generierten Referenztexte, sondern eine wohlbedachte Kostenfunktion, der ein entsprechender Aufwand entgegen gebracht werden muss, d.h. die Herausforderung liegt beim RL eher darin, eine Umwelt und eine geeignete Funktion zum Belohnen und Bestrafen zu konstruieren, hier sind bspw. auch Evaluationsmetriken notwendig
	\item Rouge-Score in Python: \url{https://pypi.org/project/rouge-score/}
	\item Typisches Diagramm zur Visualisierung des Trainingsprozesses anfügen
	\item Verhalten des Modells interpretieren und Anpassungen ableiten, bspw. Exploitation wegen der Struktur der Artikel nochmal aufgreifen
\end{itemize}
