\chapter{Abstraktiver Ansatz}
\thispagestyle{fancy}
\label{chap:Abstraktiver Ansatz}

\noindent
Unter Kenntnis der Grundlagen des \ac{DL} und des \ac{NLP} wird nun eine Architektur konzipiert und beschrieben, welche die \ac{ATS} gemäß des abstraktiven Ansatzes ermöglicht. Hierfür werden verschiedene Experimente durchgeführt, deren Training stets auf der beschriebenen Datengrundlage erfolgt.\\


\section{Metriken}
\noindent
Zuvor sind ausgewählte Metriken offenzulegen, mit denen die Qualität der \ac{ATS} gemessen werden kann: \ac{ROUGE} und \ac{BLEU}. In der Wissenschaft werden \ac{ATS}-Modelle meist mithilfe des \ac{ROUGE}-Scores evaluiert und verglichen. Dabei erfordern diese Metriken eine \ac{REFZ} zu jeder \ac{SYSZ}, welche maschinell generiert wurde. Dies ist unabhängig davon, ob das Training überwacht oder unüberwacht durchgeführt wird. Allgemein unterliegen diese Metriken der Herausforderung, dass es für einen gegebenen Text keine objektiv beste Zusammenfassung gibt. Folglich können verschiedene \ac{SYSZ} oder sogar \ac{REFZ} gleich gut sein. Dies ist statistisch sehr schwer zu bewerten, zumal selbst Menschen aufgrund ihrer subjektiven Bewertungsweise nicht einheitlich definieren können, welche Faktoren für eine gute Zusammenfassung stehen. Weiterhin werden in den Metriken menschliche Bewertungsfaktoren wie beispielsweise Lesbarkeit nicht berücksichtigt \cite{LEM20}.\\


\subsection{ROUGE}
\noindent
\ac{ROUGE} kann zunächst mithilfe folgender Kennzahlen weitergehend differenziert werden: Recall, Precision und Measure. Dabei quantifiziert der Recall-Score den Anteil der sowohl in \ac{REFZ} als auch in \ac{SYSZ} vorkommender Wörter gemäß folgender Formel:\\
$$\frac{\text{Anzahl übereinstimmender Wörter}}{\text{Anzahl der Wörter in der REFZ}}$$
\newpage

\noindent
Sei hierfür verkürzt \glqq Der Sommer war sehr warm\grqq{} als \ac{REFZ} und \glqq Der Sommer war wieder sehr warm\grqq{} als \ac{SYSZ} gegeben. Dann gilt: $\text{Recall} = \frac{5}{5} = 1.0$. Trotzdem sollen \ac{SYSZ} nicht unendlich lang werden, um alle Wörter der \ac{REFZ} abzudecken, sondern weiterhin den eigentlichen Sinn einer Zusammenfassung erfüllen. Hier quantifiziert der Precision-Score den Anteil der tatsächlich relevanten Wörter gemäß folgender Formel:\\
$$\frac{\text{Anzahl übereinstimmender Wörter}}{\text{Anzahl der Wörter in der SYSZ}}$$ \newline

\noindent
Wie man sieht, ändert sich nur der Nenner. Im oben genannten Beispiel gilt somit: $\text{Precision} = \frac{5}{6} = 0.8\overline{3}$. Nimmt man ferner \glqq Der letzte Sommer war wieder sehr warm und trocken\grqq{} als neue \ac{SYSZ} an, dann reduziert sich der Precision-Score aufgrund der erhöhten Anzahl unrelevanter Wörter wie folgt: $\text{Precision} = \frac{5}{9} = 0.5\overline{5}$. Weiterhin kann der Measure-Score als gewöhnlicher F-Score verstanden und interpretiert werden. Er ergibt sich als harmonisches Mittel zwischen dem Recall und der Precision, womit er beide Scores berücksichtigt \cite[S.~1-3]{LIN04}.\\

\noindent
\ac{ROUGE} wird allgemein auch als \ac{ROUGE}-N geschrieben, wobei das N bestimmt, ob obige Kennzahlen auf Grundlagen von Uni-, Bi- oder Trigrammen berechnet werden sollen. Im genannten Beispiel wurden also die \ac{ROUGE}-1 Recall-, Precision- und Measure-Scores berechnet. Zudem existieren Ansätze, welche die \ac{LCS} verfolgen. Diese werden hier jedoch vernachlässigt.\\

\noindent
Trotz oder gerade wegen der wissenschaftlichen Verbreitung des \ac{ROUGE}-Scores kommt immer mehr Kritik auf. Demnach kann \ac{ROUGE} beispielsweise nicht zwischen verschiedenen aber bedeutungsähnlichen Wörtern unterscheiden. Dies führt tendenziell zu einer schlechteren Bewertung, obgleich ein gegebener Text etwa in einer entsprechenden \ac{SYSZ} präzise zusammengefasst wurde. Außerdem wird den Texten zur Berechnung des \ac{ROUGE}-Scores Kleinschreibung abverlangt, unabhängig von den vorgeschalteten Modellen. Die Bewertung geschieht also eher auf syntaktischer als auf semantischer Basis. Aufgrund der bereits genannten weitreichenden Nutzung des \ac{ROUGE}-Scores und der damit gegebenen Vergleichbarkeit kommt er trotzdem in dieser Arbeit zum Einsatz \cite[S.~5]{LIN04}.
\newpage

	
\subsection{BLEU}
\noindent
\ac{BLEU} kommt der Funktionsweise von \ac{ROUGE} weitestgehend gleich. Demnach repräsentiert der Score ebenfalls die Ähnlichkeit längendefinierter N-Gramme. Dabei wird weiterhin für jede \ac{SYSZ} eine entsprechende \ac{REFZ} gefordert. Beide Metriken funktionieren indes sprachunabhängig. Im Unterschied zu \ac{ROUGE} führt \ac{BLEU} einen multiplikativen Bestrafungsterm ein, um zu kurze \ac{SYSZ} zu entwerten. Dies ist nicht notwendig, wenn die \ac{SYSZ} länger ist als die \ac{REFZ}. Der Precision-Score berücksichtigt dies bereits \cite[S.~5]{PAP02}.\\

\noindent
Obgleich der \ac{BLEU}-Score primär die Bewertung von Übersetzungen unterstützt, eignet er sich gewissermaßen auch für \ac{ATS}-Aufgaben. Zuletzt konnte wissenschaftlich bewiesen werden, dass der \ac{BLEU}-Score recht gut mit menschlichen Bewertungen korreliert. Aufgrund ähnelnder Nachteile zu denen des \ac{ROUGE}-Score und fehlender wissenschaftlicher Vergleichbarkeit wird der \ac{BLEU}-Score in dieser Arbeit nicht verwendet \cite[S.~6-7]{PAP02}.


\section{Architektur}
\noindent
Die Architektur ist als Sequence-to-Sequence-Transformer-Modell zu verstehen. Dabei wird sowohl der Encoder als auch der Decoder durch ein eigenständiges gemäß \ac{TL} vortrainiertes Modell repräsentiert. Aufgrund der bereits beschriebenen Grundlagen wird nun die weitergehende Konfiguration der Architektur definiert. Hierbei gibt es unter anderem die folgenden beiden Möglichkeiten, um den Encoder zum \ac{NLU} und den Decoder zur \ac{NLG} zu initialisieren \cite[S.~2]{ROT20}.\\

\noindent
Einerseits ist es möglich, den Encoder und den Decoder jeweils mit einem autarken Modell zu initialisieren, beispielsweise den Encoder mit \ac{BERT} und den Decoder mit \ac{GPT}. Andererseits ist es aber auch möglich, sowohl den Encoder als auch den Decoder mit dem gleichen Checkpoint eines Modells zu initialisieren, welches ursprünglich nur als Encoder trainiert wurde, beispielsweise mit \ac{BERT}. Gemäß \ac{TL} wird so im Kontext von \ac{NLP} und \ac{ATS} das Neuerlernen einer Sprache umgangen. Aufgrund der Verfügbarkeit von \ac{BERT} wird in der Folge nur letztere Möglichkeit betrachtet. Die folgenden Ausführungen sind in \autoref{pic:EncoderDecoderBert} visualisiert \cite[S.~2-3]{ROT20}.
\newpage

\begin{figure}[h!]
  \centering
  \fbox{\includegraphics[width=0.8\linewidth]{./source/images/encoderdecoderbert.png}}
  \caption{Sequence-to-Sequence-Transformer-Modell mit BERT \cite{VON20}.}
  \label{pic:EncoderDecoderBert}
\end{figure}

\noindent
Die Gewichte der bidirektionalen Self-Attention-Schichten und der Feed-Forward-Schichten aller Encoder-Blöcke werden mit den vortrainierten Gewichten von \ac{BERT} initialisiert. Dabei kann der Encoder schlichtweg als \ac{BERT} in seiner Reinform verstanden werden. Der Decoder hingegen bedarf mindestens der nachstehenden Anpassungen \cite{VON20}.\\

\noindent
Zunächst werden sogenannte Cross-Attention-Schichten zwischen den Self-Attention-Schichten und den Feed-Forward-Schichten aller \ac{BERT}-Blöcke eingeführt, um die kontextbasierten Sequenzen verarbeiten zu können. Die Gewichte der Cross-Attention-Schichten werden hierbei zufällig initialisiert \cite{VON20}.
\newpage

\noindent
Zudem werden die bidirektionalen Self-Attention-Schichten zu unidirektionalen Self-Attention-Schichten transformiert, um der autoregressiven Funktionsweise eines Decoders gerecht zu werden. Bei der autoregressiven \ac{NLG} wird angenommen, dass die Wahrscheinlichkeitsverteilung einer Sequenz in das Produkt der bedingten nächsten Wortverteilungen zerlegt werden kann \cite{VON20}. Beide Attention-Schichten basieren auf den gleichen Projektionen aus Key, Query und Value, weshalb die Gewichte dieser Schichten weiterhin mit den Werten von \ac{BERT} initialisiert werden können. Die unidirektionalen Self-Attention-Schichten berücksichtigen nun nur noch vorangegangene Token, nicht mehr auch die nachstehenden Token. Dies führt zu veränderten Ausgabevektoren im Vergleich zum ursprünglich \ac{BERT}, obwohl sie die gleichen Gewichte teilen \cite[S.~2]{ROT20}.\\

\noindent
Zuletzt wird dem Decoder eine sogenannte Language-Model-Head-Schicht hinzugefügt, dessen Gewichte mit denen des gewählten Word Embeddings initialisiert werden. Hierbei handelt es sich erneut um \ac{BERT}. Es wird deutlich, dass sich der Encoder und der Decoder viele Gewichte teilen können. Dies führt zu einer erheblichen Reduktion des Speicherbedarfs, während die Qualität anschließender \ac{NLP}-Aufgaben nahezu unverändert bleibt \cite[S.~2]{ROT20}.\\

\noindent
Die Textinhalte der Datengrundlage bedürfen überdies keiner weitergehenden Vorverarbeitung im herkömmlichen Sinne. Diese ist bekanntermaßen sehr individuell und stark modellabhängig. Unter Verwendung der als sehr robust geltenden Transformer-Architekturen entfällt daher die sonst übliche Textbereinigung sowie die Textnormalisierung. Dies unterliegt der Annahme, dass Transformer-Architekturen potenziell aus jeder Eigenart ein relevantes Feature schaffen können, welches das spätere Ergebnis begünstigt. Von der zugeführten Interpunktion und den vielfältigen Wortformen wird sich indes erhofft, potenzielle Mehr- oder Uneindeutigkeiten zu minimieren. Das Fine-Tuning sollte darüber hinaus stets unter gleichen Bedingungen wie das initiale Training stattfinden. Gleichzeitig sinkt hierdurch der vorverarbeitende Aufwand und damit auch etwaige Wartezeiten bei der praktischen Anwendung bereits trainierter Modelle in Echtzeit. Dennoch ist es möglich, bestimmte Vorverarbeitungsschritte a posteriori zu implementieren. Die Auswirkungen auf das Modell und die entsprechenden Ergebnisse würden somit zugleich messbar.
\newpage

\noindent
In der sonstigen technischen Vorbereitung ist weiterhin ein Tokenizer zu definieren. Dieser entstammt ebenfalls \ac{BERT} und berücksichtigt Groß- und Kleinschreibung. \ac{BERT} kann Sequenzen bis zu einer maximalen Länge von 512 Token verarbeiten. Dies unterschreitet zwar die durchschnittliche Textlänge der beschriebenen Korpora, kann jedoch unter der Annahme, dass wichtige Informationen zumeist am Anfang von Texten stehen, akzeptiert werden \cite{VON20}.\\

\noindent
Von einer schlichten Erhöhung der maximalen Tokenlänge ist indes abzuraten, da hierbei ein quadratischer Anstieg der Rechenzeit und des Speicherbedarfs zu erwarten ist. Zudem wurde \ac{BERT} ausschließlich auf Texten mit einer maximalen Tokenlänge von 512 trainiert. Ein denkbarer Lösungsansatz, welcher an dieser Stelle nur genannt sei, ist der sogenannte Sliding-Window-Approach. Hierbei besteht jedoch die Gefahr, dass langfristige Abhängigkeiten verloren gehen. Zuvor sind außerdem entsprechende Testläufe durchzuführen. Weiterhin existieren Ansätze wie etwa Longformer oder auch Big Bird, welche die Verarbeitung langer Sequenzen verfolgen \cite{ZAH21}. Diese versuchen zugleich, lineare Komplexität zu erreichen, beispielsweise mithilfe lokaler Attention-Mechanismen, die wiederum mit globaler Attention verknüpft sind \cite{BEL20}. Ein eher anwendungsbezogener Workaround besteht hingegen darin, den jeweils eingehenden Rohtext alle 512 Token zu unterteilen, die Subtexte einzeln zusammenzufassen und die Zusammenfassungen zu konkatenieren. Dies betrifft jedoch nicht das eigentliche Training \cite[S.~2]{DIN20}.\\

\noindent
Texte werden also zusammenfassend in den nachfolgenden Schritten jeweils nach 512 Token abgeschnitten. Die maximale Tokenlänge der entstehenden Zusammenfassungen wird auf 128 limitiert. Anpassungen, welche etwa im Laufe der experimentgetriebenen Entwicklung und Optimierung an Potenzial gewinnen, werden an den entsprechenden Stellen ergründet und evaluiert.


\section{Experimente}
\noindent
Die Entwicklung und die Durchführung aller Experimente geschieht in Python. Dies ist eine Programmiersprache, welche sich insbesondere für \ac{ML}- und \ac{DL}-Zwecke eignet. Dabei werden Trainingsprozesse durch \ac{CUDA} unterstützt, wenn entsprechende Voraussetzungen erfüllt sind. \ac{CUDA} ist eine von NVIDIA entwickelte Technik, welche es ermöglicht, bestimmte Operationen mithilfe der GPU zu beschleunigen \cite{NVI21}. Zudem ist es in dieser Umgebung möglich, vortrainierte Modelle wie \ac{BERT} zu laden und Architekturen weitergehend gemäß der oben definierten Konfiguration zu präparieren, darunter beispielsweise die beschriebene Encoder-Decoder-Architektur. Dies wird durch die Bibliothek PyTorch und das US-Unternehmen HuggingFace, welches den Code als Open Source bereitstellt, ermöglicht. HuggingFace stellt zudem verschiedene Klassen zum Trainieren von Sequence-to-Sequence-Modellen bereit \cite{HUG21}. Darüber hinaus erfolgen alle Experimente dieser Arbeit über einen legitimierten Zugang auf dem Hochleitungsrechner der TU Dresden, namentlich Taurus, um das Potenzial der verfügbaren Umgebung mithilfe einer leistungsstarken GPU (NVIDIA V100) vollends auszuschöpfen \cite{ZIH21}. Der Quellcode ist dem Anhang zu entnehmen. Experimente umfassen indes stets die folgenden Schritte: Initialisierung, Training, Evaluation.\\

\noindent
Zunächst erfolgt die Reproduktion des \ac{SOTA}, um eine Baseline zu setzen, an welcher sich in nachfolgenden Experimenten verglichen und gemessen werden kann. Daher ist es unabdingbar, ein erstes Modell auf dem englischen Korpus zu trainieren. Dies folgt der beschriebenen Architektur und der entsprechenden Konfiguration ohne Kompromisse. Das entsprechende Training verlief komplikationslos und mit der erwarteten Fehlerabnahme. Letztlich sind die folgenden \ac{ROUGE}-Scores zu verzeichnen: R-Recall: 15.78, R-Precision: 10.25, R-Measure: 12.11. Diese stimmen unter Akzeptanz womöglich unbekannter Nebenbedingungen mit den umworbenen \ac{ROUGE}-Scores des \ac{SOTA} überein.\\

\noindent
Zudem wird der in Anhang A befindliche englischsprachige Text mithilfe des trainierten Modells zusammengefasst. Der Text unterscheidet sich strukturell von den eingegangenen Trainingsdaten und weist etwas über 1.000 Token auf. Weiterhin entstammt er einer dem Modell unbekannten Domäne. Fachlich ist er zugleich nicht zu spezifisch. Er handelt von den Anschlägen des 11. Septembers 2001 in New York. Die entstehende Zusammenfassung ist nachfolgend einzusehen. Es ist aus menschlich subjektiver Sicht erkennbar, dass nicht nur die verdichteten Informationen, sondern auch die Grammatik und die Orthographie weitestgehend korrekt sind.\\

\noindent\fbox{%
\parbox{\textwidth}{%
Firefighters, police officers, search-and-rescue dogs headed to Ground Zero to search for survivors. They didn't know how many people were trapped alive in the wreckage. They were able to search through the unstable piles of rubble for air pockets. By September 11th, workers had rescued all the people trapped at the site.
}%
}\\
